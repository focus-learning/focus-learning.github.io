\documentclass[l1pt]{article}


\usepackage[english]{babel}
\usepackage{amsmath,amsthm,amssymb}
\usepackage{mdframed}
\usepackage{enumerate}
\usepackage{varwidth}

\setlength\parindent{0pt}
\setlength{\parskip}{0.6em}

%\usepackage[]{microtype}



%% Sets page size and margins
\usepackage[margin=1in]{geometry}
\linespread{1.1}

%% Useful packages
\usepackage[dvipsnames]{xcolor}
\usepackage{fancyhdr}
\pagestyle{fancy}
\lhead{Vieta's Formulae}
\rhead{\thepage}
\usepackage{graphicx}
\usepackage{dirtytalk}
\usepackage{hyperref}
\hypersetup{colorlinks=true, linkcolor=MidnightBlue, urlcolor=BrickRed, pdftitle=Vieta's Formulae, pdfpagemode=FullScreen}


\theoremstyle{plain}
\newtheorem{thm}{Theorem}[section]
\newtheorem{prop}[thm]{Proposition}
\newtheorem{lemma}[thm]{Lemma}
\newtheorem{corollary}[thm]{Corollary}

\theoremstyle{definition}
\newtheorem{example}[thm]{Example}
\newtheorem{exercise}{Exercise}[section]

\theoremstyle{remark}
\newtheorem*{remark}{Remark}
\newtheorem*{solution}{Solution}

\title{\textbf{Vieta's Formulae}}
\author{\textbf{Perryn Chang}}
\date{\today}

\begin{document}
\maketitle

\begin{abstract}
In this handout, we go over \textbf{Vieta's Formulae}, which provide relationships between the roots and coefficients of polynomials.
\end{abstract}

\tableofcontents

\eject

\section{Introduction}

Suppose that we have the polynomial
\[P(x)=a_nx^n+a_{n-1}x^{n-1}+\dots+a_0,\] and we had to relate the roots of the polynomial to the coefficients. Looking immediately, it seems impossible, since we are not given any information on the roots of $P$. However, once we write the polynomial in factored form, the roots come out:
\[P(x)=a_n(x-r_1)(x-r_2)\dots(x-r_n).\] With these 2 equations of $P$, we can now relate the roots of $P$ to coefficients of $P$. (Try it yourself?!)

In the next 2 sections, we will dive deeper into how the roots are related to the coefficients. The equations this handout will go over are called \textbf{Vieta's Formulae}, named after French Mathematician Fran\c{c}ois Vi\`{e}te.


\section{Vieta's in Quadratics}

We begin with the most basic theorem of Vieta's Formulae: when it is used on a quadratic. Using this idea, we will move on to more challenging problems.

\begin{thm}
\label{thm:easyquadratic}
\begin{mdframed}
Consider a quadratic polynomial \[P(x)=ax^2+bx+c.\] Then, the sum of the two roots of $P$ is $-\frac{b}{a}$ and the product of the two roots is $\frac{c}{a}$. In other words, if the two not necessarily distinct roots of $P$ are $r_1$ and $r_2$, then \[r_1+r_2=-\frac{b}{a}\] and \[r_1 r_2=\frac{c}{a}.\]
\end{mdframed}
\end{thm}

\begin{proof}
Let the two roots of $P$ be $r_1$ and $r_2$. By the Fundamental Theorems of Algebra, $P$ can be written as \[a(x-r_1)(x-r_2).\] Expanding this, we get \[a(x^2-(r_1+r_2)x+r_1 r_2) \implies ax^2-a(r_1+r_2)x+ar_1 r_2.\] Since two polynomials are equal if and only if their coefficients are equal, we can equate the coefficients or $ax^2+bx+c$ and $ax^2-a(r_1+r_2)x+ar_1 r_2$. We have

\begin{equation}  \label{eq:quadratic1} b=-a(r_2+r_2) \end{equation} and
\begin{equation}  \label{eq:quadratic2}  c=ar_1 r_2.  \end{equation}
Dividing Equation \ref{eq:quadratic1} by $-a$ and Equation \ref{eq:quadratic2} by $a$, we get $r_1+r_2=-\frac{b}{a}$ and $r_1 r_2=\frac{c}{a}$, as desired.

\end{proof}

An important thing to realize here is that our proof did not rely on the fact that any of the variables were real numbers. This means that Vieta's Formulae works for all complex numbers. When you are solving problems and it asks you to find the sum of the real roots, you may \textbf{not} directly use Vieta's Formulae. Of course, you may use Vieta's if you know all the roots are real.

\bigskip

In most problems, the solution typically involves manipulating the equations that we derive from Vieta's Formulae. If the problem asks for a relationship between the roots of a polynomial, it is typically a hint that Vieta's Formulae should come into use. We will see this in the first example.

\begin{example}
Given that $x_1$ and $x_2$ are the two roots of the equation $4x^2+8x+27$, find the value of $x_1 ^{2}+x_2 ^{2}.$
\end{example}

\begin{solution}
Although we could find the value of the 2 roots and substitute them into the expression, it might take a while. Also, since the problem specifically asked us to find $x_1 ^{2}+x_2 ^{2}$ rather than simply $x_1$ and $x_2$, it probably does not want us to find the two roots directly.

Using Vieta's, we get that $x_1+x_2=-\frac{8}{4}=-2$ and $x_1 x_2=\frac{27}{4}$. Now, we just have to manipulate these equations until we find the result we desire. Squaring the first equation, we get that \[x_1 ^{2}+2x_1x_2+x_2^{2}=4.\] Multiplying the second equation by 2, we get $2x_1x_2=\frac{27}{2}$. Subtracting this equation from $x_1 ^{2}+2x_1x_2+x_2^{2}=4$, we get $x_1^{2}+x_2^{2}=-\frac{19}{2}$, which is the equation we want. Our answer is $-\frac{19}{2}$.
\end{solution}

In many cases, Vieta's Formulae can help us solve a problem, but will typically not completely find the answer for us. Remember to use other algebraic strategies from our toolbox.

\begin{example}[2020 AOIME]
Let $P(x) = x^2 - 3x - 7$, and let $Q(x)$ and $R(x)$ be two quadratic polynomials also with the coefficient of $x^2$ equal to $1$. David computes each of the three sums $P + Q$, $P + R$, and $Q + R$ and is surprised to find that each pair of these sums has a common root, and these three common roots are distinct. If $Q(0) = 2$, then $R(0) = \dfrac mn$, where $m$ and $n$ are relatively prime positive integers. Find $m+n$.
\end{example}

\begin{solution}
Since we are given that $Q$ and $R$ are quadratic polynomials with the coefficient of $x$ equal to 1, we can let \[Q(x)=x^2+ax+b\] and \[R(x)=x^2+cx+d.\] Since $R(0)=d$, our goal is to find the value of $D$. However, since we know that $Q(0)=2$, $b=2$. Hence, we can write \[Q(x)=x^2+ax+2.\] Since the problem asks us to relate the sum of each pair of polynomials, we compute that  \[(P+Q)(x)=2x^2+(a-3)x-5,\] \[(P+R)(x)=2x^2+(c-3)x+d-7,\] and \[(Q+R)(x)=2x^2+(c+a)x+d+2.\] Now, we are also given that each pair of these sums has a common root, so we can let the roots of $(P+Q)$ be $r$ and $s$, the roots of $(P+R)$ be $s$ and $t$, and the roots of $(Q+R)$ be $r$ and $t$. We can now proceed using Vieta's Formulae. We have the equations \[r+s=\frac{3-a}{2}, s+t=\frac{3-c}{2}, r+t=\frac{-a-c}{2}, rs=-\frac{5}{2}, st=\frac{d-7}{2}, rt=\frac{d+2}{2}.\] Adding the first two equations, we get that \[r+2s+t=\frac{6-a-c}{2}.\]Subtracting the third equation, we get \[2s=3 \implies s=\frac{3}{2}.\] Using the fourth equation, we get that \[\frac{3}{2}r=-\frac{5}{2} \implies r=-\frac{5}{3}.\] Since we want to find the value of $d$, we can create a system of equations using the fifth and sixth equations. We have that \[\frac{3}{2}t=\frac{d-7}{2}\] and \[-\frac{5}{3}t=\frac{d+2}{2}.\]Solving the system of equations, we find that  $d=\frac{52}{19}$, so our final answer is $71.$


\end{solution}

\bigskip

In this problem, we had to first use Vieta's Formulae to create a bunch of equations. Within the equations, we had to choose the ones that would help us find our answer. Although in this handout I knew which equations would be the easiest and quickest in determining the answer, in most scenarios some experimenting might be needed before you see which equations to use. Also, keep in mind that there are other ways to find the answer using the 6 equations as well.





\section{Vieta's in Higher Degrees}

We are not solely limited to using Vieta's Formulae with quadratics. Using the same idea as the quadratic version, we can expand the formula to higher degrees.

\begin{thm}
\begin{mdframed}
Consider the polynomial \[P(x)=a_nx^n+a_{n-1}x^{n-1}+a_{n-2}x^{n-2}+\dots+a_1x+a_0.\] Let the $n$ roots of this polynomial be \[r_1, r_2, r_3, \dots, r_n.\] Then, we have the equations
\[r_1+r_2+\dots+r_n=-\frac{a_{n-1}}{a_n}\]
\[(r_1r_2+r_1r_3+r_1r_4+\dots+r_1r_n)+(r_2r_3+r_2r_4+\dots+r_2r_n)+\dots(r_{n-1}r_n)=\frac{a_{n-2}}{a_n}\] \[\vdots\] \[r_1r_2\ \dots r_n=(-1)^n\frac{a_0}{a_n}\]

\end{mdframed}
\end{thm}

\begin{proof}
We can write $P(x)$ as \[P(x)=a_n(x-r_1)(x-r_2)\dots(x-r_n).\] Expanding the right side, we get that it is
\[a_nx^n-a_n(r_1+r_2+\dots+r_n)x^{n-1}+a_n(r_1r_2+r_1r_3+\dots+r_{n-1}r_n)x^{n-2}+\dots+(-1)^na_nr_1r_2\dots r_n.\] Comparing coefficients, we have
\[-a_n(r_1+r_2+\dots+r_n)=a_{n-1} \implies r_1+r_2+\dots+r_n=-\frac{a_{n-1}}{a_n}\]
\[a_n(r_1r2+r_1+r_3+\dots+r_{n-1}r_n)=a_{n-2} \implies r_1r_2+r_1r_3+\dots+r_{n-1}r_n=\frac{a_{n-2}}{a_n}\]
\[\vdots\]
\[(-1)^na_nr_1r_2\dots r_n=a_0 \implies r_1r_2\dots r_n=(-1)^n\frac{a_0}{a_n}.\]
\end{proof}

Again, I would like to note that Vieta's by itself is usually not enough to completely solve a problem. Other (algebraic) techniques should still be used in order to find an answer.

\begin{example}[Brilliant]
Let $r_1, r_2, r_3$ be the roots of the polynomial $5x^3-11x^2+7x+3.$ Evaluate
\[r_1(1+r_2+r_3)+r_2(1+r_3+r_1)+r_3(1+r_1+r_2).\]
\end{example}

\begin{solution}
Looking at the equation, there seems to be no way to use Vieta's immediately. We can start by expanding the expression. We get \[r_1+r_1r_2+r_1r_3+r_2+r_2r_3+r_2r_1+r_3+r_3r_1+r_3r_2.\] This can be written as \[r_1+r_2+r_3+2(r_1r_2+r_2r_3+r_1r_3).\] Now, we have an expression that we can simplify using Vieta's. By Vieta's, we have $r_1+r_2+r_3=\frac{11}{5}$ and $r_1r_2+r_2r_3+r_3r_1=\frac{7}{3}$, so our answer is $\frac{11}{5}+2(\frac{7}{5})=5$.
 \end{solution}

 We solved this problem by manipulating an expression until we were able to substitute things in using Vieta's. This is a common strategy that occurs in problems involving Vieta's Formulae.

 \begin{example}[Brilliant]
 Find all triples of complex numbers which satisfy the following system of equations:
 \[a+b+c=0\] \[ab+bc+ca=0\] \[abc=0.\]
 \end{example}

 \begin{solution}
 Without seeing a polynomial, it would appear quite difficult to relate this problem to Vieta's Formulae. However, the equations we are given are in the form of Vieta's Formulae. This means that we can create a polynomial, with $a, b, c$ being the roots of our polynomial. Since there are 3 roots, the polynomial must be cubic. Therefore, for simplicity, we can choose the monic\footnote{Note that choosing a monic polynomial does not affect the roots, since the coefficient of the leading term can just be factored out. If you do not see this, try writing the polynomial in factored form.} polynomial \[P(x)=x^3+mx^2+nx+p.\] We know that $a+b+c=-m=0$, $ab+bc+ca=n=0$, and $abc=-p=0$, so our polynomial becomes \[P(x)=x^3.\] It is easy to see that the only root of $P$ is 0 (with a multiplicity of 3). Since we let $a, b, c,$ be the roots of this polynomial, the values of $a, b, c$ must be all 0.
 \end{solution}

 In this problem, rather than getting a polynomial and using Vieta's on it, we used it \say{in reverse}. Although these problems are not common, keep an eye out for when you see equations that satisfy Vieta's Formulae.

\begin{example}[2019 AIME]
For distinct complex numbers $z_1,z_2,\dots,z_{673}$, the polynomial
\[ (x-z_1)^3(x-z_2)^3 \cdots (x-z_{673})^3 \]can be expressed as $x^{2019} + 20x^{2018} + 19x^{2017}+g(x)$, where $g(x)$ is a polynomial with complex coefficients and with degree at most $2016$. The value of
\[ \left| \sum_{1 \le j <k \le 673} z_jz_k \right| \]can be expressed in the form $\tfrac{m}{n}$, where $m$ and $n$ are relatively prime positive integers. Find $m+n$.
\end{example}

\begin{solution}
First, please make sure you know what the problem is asking before moving on to reading the solution. The notation used in this problem is rather confusing.

Notice that we are given only the first few terms of the polynomial. This seems like it would be a hint to use Vieta's.

First, note that from the factored form of the polynomial, we can immediately observe that the roots are \[z_1, z_1, z_1, z_2, z_2, z_2, \dots, z_{673}, z_{673}, z_{673}, \]where each $z_i$ has a multiplicity of 3. From Vieta's, we have \[3(z_1+z_2+\dots+z_{673})=-20\] and

\begin{equation} \label{eq:3}
3(z_{1}^2+z_{2}^2+\dots+z_{673}^2)+9((z_1z_2+z_1z_3+\dots+z_1z_{673})+(z_2z_3+z_2z_4+\dots+z_2z_{673})+(z_3z_4+\dots+z_3z_{673})+\dots+z_{672}z_{673})=19.
\end{equation}

Notice the $9(\text{something})$ term of the second equation is what we need to find, so if we can find the value of $3(z_1^2+z_2^2+\dots+z_{673}^2),$ then we would basically be done. Dividing the first equation by 3, we get \[z_1+z_2+\dots +z_{673}=-\frac{20}{3}.\] Squaring this, we get \[(z_1^2+z_2^2+\dots+z_{673}^2)+2((z_1z_2+z_1z_3+\dots+z_1z_{673})+(z_2z_3+\dots+z_2z_{673})+\dots+z_{672}z_{673})=\frac{400}{9}.\] Now, observe that in this equation, we have $(z_1z_2+z_1z_3+\dots+z_1z_{673})+(z_2z_3+\dots+z_2z_{673})+\dots+z_{672}z_{673}$ as well, which is the value the question is asking for. We can write the equation as \[3(z_1^2+z_2^2+\dots+z_{673}^3)=\frac{400}{3}-6((z_1z_2+z_1z_3+\dots+z_1z_{673})+(z_2z_3+\dots+z_2z_{673})+\dots+z_{672}z_{673}).\] Substituting into \ref{eq:3},
\[3((z_1z_2+z_1z_3+\dots+z_1z_{673})+(z_2z_3+\dots+z_2z_{673})+z_{672}z_{673})=19-\frac{400}{3}.\] Simplifying the right hand side and dividing by 3, we get that \[(z_1z_2+z_1z_3+\dots+z_1z_{673})+(z_2z_3+\dots+z_2z_{673})+z_{672}z_{673}=-\frac{343}{9},\] so the answer is $343+9=352.$


\end{solution}



\section{Problems}

\begin{exercise}
Determine the sum and product of the roots of
\[(a) \qquad 3x^2-8x+13\]
\[(b) \qquad 4x^4-5x^2+1\]
\[(c) \qquad x^5-5x^4+9x^3-2x+1\]
\end{exercise}

\begin{exercise}
Given the polynomial $P(x)=5x^3+8x^2-4x-1$ with roots $a, b, c$, find the value of \[\frac{1}{a}+\frac{1}{b}+\frac{1}{c}.\]
\end{exercise}

\begin{exercise}
Complete the proof of Vieta's for cubic polynomials.
\end{exercise}

\begin{exercise}[AoPS]
Let $p$ and $q$ be the roots of $4x^2-15x-12$. Compute $(p-5)(q-5).$
\end{exercise}

\begin{exercise}
Find the values of $a$ and $b$ that satisfy $a+b=8$ and $ab=15$.
\end{exercise}

\begin{exercise}
The roots of the polynomial $x^3+ax^2-3x-b$ are $3, -5, -6$. Find the value of $a$ and $b$.
\end{exercise}

\begin{exercise}
Let $r, s, t$ be the roots of the equation $x^3-5x^2+x-3$. Compute $r^3st+rs^3t+rst^3+2r^2s^2t+2rs^2t^2+2r^2st^2$.
\end{exercise}

\begin{exercise}[Brilliant]
Suppose $k$ is a number such that the cubic polynomial $P(x)=-2x^3+48x^2+k$ has three integer roots that are all prime numbers. How many possible distinct values are there for $k$?
\end{exercise}

\begin{exercise}[OTSS Season 1]
Given the polynomial $x^3 + 6 x^2 + 5x - 7$ with roots $r_1$, $r_2$, and $r_3$, what is the value of\[\left(\frac{1}{r_1}+\frac{1}{r_2}+\frac{1}{r_3}\right)\cdot (r_1^2-3r_1+2)\cdot (r_2^2-3r_2+2)\cdot (r_3^2-3r_3+2)?\]
\end{exercise}

\begin{exercise}[2015 AIME]
Steve says to Jon, "I am thinking of a polynomial whose roots are all positive integers. The polynomial has the form $P(x)=2x^3-2ax^2+(a^2-81)x-c$ for some positive integers $a$ and $c$. Can you tell me the values of $a$ and $c$?"

After some calculations, Jon says, "There is more than one such polynomial."

Steve says, "You’re right. Here is the value of $a$." He writes down a positive integer and asks, "Can you tell me the value of $c$?"

Jon says, "There are still two possible values of $c$."

Find the sum of the two possible values of $c$.
\end{exercise}

\begin{exercise}[Brilliant]
If $\alpha$ and $\beta$ are the roots to the equation $x^2+2x+3=0$, what is the monic quadratic equation whose roots are $\left(\alpha-\frac{1}{\alpha}\right)^2$ and $\left(\beta-\frac{1}{\beta}\right)^2$.
\end{exercise}


\end{document}
