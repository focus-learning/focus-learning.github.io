\documentclass[l1pt]{article}


\usepackage[english]{babel}
\usepackage{amsmath,amsthm,amssymb}
\usepackage{mdframed}
\usepackage{enumerate}

%\usepackage[]{microtype}




%% Sets page size and margins
\usepackage[margin=1in]{geometry}
\linespread{1.1}

%% Useful packages
\usepackage[dvipsnames]{xcolor}
\usepackage{fancyhdr}
\pagestyle{fancy}
\lhead{Introduction to the Pigeonhole Principle}
\rhead{\thepage}
\usepackage{graphicx}
\usepackage{dirtytalk}
\usepackage{hyperref}
\hypersetup{colorlinks=true, linkcolor=MidnightBlue, pdftitle=Introduction to the Pigeonhole Principle, pdfpagemode=FullScreen}


\theoremstyle{plain}
\newtheorem{thm}{Theorem}[section]
\newtheorem{prop}[thm]{Proposition}
\newtheorem{lemma}[thm]{Lemma}
\newtheorem{corollary}[thm]{Corollary}

\theoremstyle{definition}
\newtheorem{example}[thm]{Example}
\newtheorem{exercise}{Exercise}[section]

\theoremstyle{remark}
\newtheorem*{remark}{Remark}
\newtheorem*{solution}{Solution}


\title{\textbf{Introduction to the Pigeonhole Principle}}
\author{\textbf{Perryn Chang}}
\date{\today}

\begin{document}
\maketitle

\begin{abstract}
The goal of this handout is to let the reader become familiar with the Pigeonhole Principle, an idea used in all levels of competitive mathematics. The Pigeonhole Principle is also known as \textbf{Dirichlet's Box Principle}.
\end{abstract}

\tableofcontents

\eject

\section{Introduction}
Going with the spirit of the name of the principle, consider $8$ pigeons and $7$ pigeonholes. It is not hard to realize that if all the pigeons went to a pigeonhole, at least one hole would have more than one pigeon. We can easily generalize this.

\begin{mdframed}
    \begin{prop}
    \label{prop:beginintro}
    If $n$ objects are placed into $k$ boxes, where $n>k$, then at least $1$ box will have more than $1$ object.
    \end{prop}
\end{mdframed}

Now, what if I had 16 pigeons and 7 pigeonholes. In this case, at least one pigeonhole would contain at least 2 pigeons. We can create a better Pigeonhole Principle. That is\dots 


\begin{mdframed}
\begin{thm}
If there are $n$ boxes and more than $k\cdot n$ objects are being distributed among the boxes, at least one box will contain more than $k$ objects.
\end{thm}
\end{mdframed}

\begin{proof}
We proceed by contradiction. Assume that all of the $n$ boxes contain $k$ or less objects. Then, there are a maximum of $k\cdot n$ objects, which contradicts the assumption that there are more than $k\cdot n$ objects, so we are done.
\end{proof}


Using this idea, we can we can update Proposition \ref{prop:beginintro}.

\begin{mdframed}
    \begin{prop}
    If $n$ objects are placed into $k$ boxes, where $n>k$, then at least one box must contain $\left\lfloor \dfrac{n-1}{k} \right\rfloor + 1 $ balls.
    \end{prop}
\end{mdframed}

Although we have this formula, there is no need to memorize it. Usually, common sense will suffice when the problem requires you to use this formula. The proof is left as an exercise for the reader. For simplicity, the \say{pigeons} will be known as balls for now on and the \say{pigeonholes} will be known as boxes.

\section{Basic Examples}
Let's start with a simple example.

\begin{example}
There exists a group of 4 friends. How many apples should I give them such that at least one person will receive more than 1 apple.
\end{example}

\begin{solution}
By the Pigeonhole Principle, when there are 5 apples, at least person will receive more than one apple, so our answer is \boxed{5}.
\end{solution}

\bigskip

In some scenarios, the Pigeonhole Principle cannot be directly used, but the idea behind it is similar. Here is an example.

\begin{example}
(2019 AMC) A box contains $28$ red balls, $20$ green balls, $19$ yellow balls, $13$ blue balls, $11$ white balls, and $9$ black balls. What is the minimum number of balls that must be drawn from the box without replacement to guarantee that at least $15$ balls of a single color will be drawn$?$
\end{example}

\begin{solution}
If we had an infinite number of balls for each color, then it's easy to see that the answer would be $14\cdot 6+1=85$, since in the worst case scenario we have to draw 14 balls of each color. This means that for the colors that have less than 15 balls, we can simply draw all of them. We can draw all the black, white, and blue balls, while drawing 14 of the yellow, green, and red balls. The next ball we draw forces some color to have 15 balls, so our answer is $\boxed{76}$.
\end{solution}

\bigskip

We are not only limited to objects as the balls and boxes; they could be anything. A classic example involves modular arithmetic.

\begin{example}
Prove that if we are given any 11 integers, there exists a pair of integers such that their difference is divisible by 10.
\end{example}

\begin{solution}
Here, the balls and boxes aren't that clear. We are working with solely the integers. However, the words \say{difference} and \say{divisible} hint at modular arithmetic. We can express two integers being congruent modulo 10 as \[a \equiv b \pmod {10}\]

Note that there are 10 choices for the residue of $a$ and $b$. But, we are given 11 total numbers. We can let our balls be the 11 numbers and the boxes be the residues. By the Pigeonhole Principle, at least 2 numbers will be congruent $\mod 10$, which implies that their difference is divisible by 10.
\end{solution}

\bigskip

Our key idea here is determining our balls in boxes. We had to cleverly figure out that using the residues was the crux in this problem. Determining what our balls and boxes are is often the most important part in Pigeonhole problems. In the next example, we will see how we can use this idea in geometry as well.

\begin{example}
Prove that for any 8 points chosen in a unit circle, there exists 2 points that are at most 1 unit away from each other.
\end{example}

\begin{solution}
We want to somehow create the boxes so that each box will have more than 1 ball. The balls seem to be the 8 points. We are able to do this by dividing the circle into 7 congruent sectors. By the Pigeonhole Principle, at least one sector will have more than 1 point. In a sector, two points can be at most 1 unit apart, so we are done \footnote{Note that this does not mean 8 is the minimum number of points such that the condition is satisfied.}.
\end{solution}

\section{More Difficult Examples}

Although in the last section the boxes weren't given to us, it was usually quite easy to identify them. Also, the number of balls was always directly to us. In more advanced problems, however, we often have to create the balls and boxes on our own. 

As usual, let's start with an example.

\subsection{Connection with the Mean and Expected Value}
We begin this subsection with two theorems.

\begin{mdframed}
    \begin{thm}
    \label{thm:arithmetic mean}
    If the arithmetic mean of a set of $n$ values is $x$, then there exists one element that is greater than or equal to $x$ and one element that is less than or equal to $x$
    \end{thm}
\end{mdframed}

\begin{proof}
We proceed by contradiction. To prove that there exists an element that is greater than or equal to $x$, we assume that all elements are less than $x$. We have \[\sum_{i=1}^n a_i=nx .\] However, all the $a_is$ are less than $x$, so the equation cannot hold, and we have a contradiction. The proof is similar for proving there exists an element that is less than or equal to $x$, so we are done.
\end{proof}

\begin{mdframed}
    \begin{thm}
    \label{thm:expected value}
    If the expected value of a variable x is $E(x)$, then there exists a case of x where x is greater than or equal to $E(x)$ and a case of x where x is less than or equal to $E(x)$.
    \end{thm}
\end{mdframed}

The proof is very similar to Theorem \ref{thm:arithmetic mean} and the theorem is very intuitive, so we will not be proving it. It will be left as an exercise if the reader wishes to prove it. 

An immediate corollary very similar to Theorem \ref{thm:expected value} follows.

\begin{mdframed}
    \begin{corollary}
    If the expected value of x is $E(x)$, and x must be an integer, then there exists a value of x that is greater than or equal to $\left \lceil{E(x)}\right \rceil$ and a value of x that is less than or equal to $\left \lfloor{E(x)}\right \rfloor $.
    \end{corollary}
\end{mdframed}

Let's see this in action.

\begin{example}

\end{example} 

\subsection{Pigeons and Points (Geometry)}

The Pigeonhole Principle is also occasionally seen with geometry. In most cases, it deals with points within a plane or space. Some clever thinking is usually the only thing required to solve these type of problems.

\begin{example}[USAMTS]
Every point in a plane is either red, green, or blue. Prove that there exists a rectangle in the plane such that all of its vertices are the same color.
\end{example}

\begin{solution}
Choose a rectangle in the plane with 4 points as the columns and 19 points as the rows. Consider the first row. Since there are four points and 3 colors, at least 2 points will be the same color. 

Now consider the 19 rows. Since for each row there are 3 ways to choose the color that is repeated and $\binom{4}{2}=6$ ways to choose which points are repeated, there are 18 possible arrangements of the 3 colors. Since there are 19 rows and 18 possible arrangements of the 3 colors, by the Pigeonhole Principle, 2 rows will be exactly the same and contain at least 2 points of the same color. The points that are the same color in those two rows will form the rectangle, and we are done.
\end{solution}

Our next one uses a bit of number theory, but again deals with points within a plane.

\begin{example}
Prove that if I choose any 5 distinct lattice points on the coordinate plane and compute the midpoints of all the segments formed by those 5 lattice points, there exists a midpoint that lies on a lattice point.
\end{example}

\begin{solution}

\end{solution}







\section{Problems}

\begin{exercise}[Classic]
If I have an infinite number of red socks, yellow socks, pink socks, green socks, and blue socks, how many socks do I have to draw I order to guarantee I have a pair of socks that are the same color.
\end{exercise}

\begin{exercise}[2006 AMC]
 Six distinct positive integers are randomly chosen between 1 and 2006, inclusive. What is the probability that some pair of these integers has a difference that is a multiple of 5?
\end{exercise}

\begin{exercise}[1972 IMO]
Prove that from a set of ten distinct two-digit numbers (in the decimal system), it is possible to select two disjoint subsets whose members have the same sum.
\end{exercise}




\end{document}
