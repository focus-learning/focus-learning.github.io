\documentclass[letterpaper]{article}

\usepackage{sectsty}
\usepackage{graphicx}
\usepackage{amsmath}
\usepackage{amsthm}
\usepackage{mdframed}
\usepackage{xcolor}
\usepackage{array}
\usepackage{asymptote}
\theoremstyle{plain}

\newtheorem{thm}{Theorem}[section]

\newtheorem{prop}[thm]{Proposition}
\newtheorem{corollary}{Corollary}[thm]
\newtheorem{lemma}[thm]{Lemma}

\theoremstyle{definition}

\newtheorem{definition}[thm]{Definition}
\newtheorem{example}[thm]{Example}
\newtheorem{exercise}[thm]{Exercise}

\theoremstyle{remark}
\newtheorem*{solution}{Solution}
\newtheorem*{remark}{Remark}


\begin{document}

\title{Basic Triangle Centers}
\maketitle

\begin{abstract}
    Triangle centers are an important facet of euclidean geometry.
    Often problems require an intricate understanding of the relationships between triangle centers.
\end{abstract}
\section{Concurrency}
    The fact that many of the following triangles exist can be surprising. We will establish a condition which will help us prove that these centers do in fact exist.
    \begin{mdframed}
        \begin{thm}[Ceva's] 
            In $\triangle ABC$, points D, E, and F are on sides BC, AC, and AB respectively. $\frac{BD}{DC}\cdot \frac{CE}{EA}\cdot\frac{AF}{FB} = 1$, if and only if the segments AD, BE, and CF are concurrent.
            
        \end{thm}
        
    \end{mdframed} 
    \begin{proof}
        We will first show that the direct implication holds. Let the point of concurrency be X. We have the following area ratio 
        $$\frac{[ABD]}{[ADC]} = \frac{[XBD]}{[XDC]}=\frac{BD}{DC}.$$
        As you may know, this implies $$\frac{[ABD-XBD]}{[ADC-XDC]}=\frac{[ABX]}{[AXC]}=\frac{BD}{DC}.$$
        Similarly, for the other sides we have 
        $$\frac{[BCE-XCE]}{[BEA-XEA]}=\frac{[BCX]}{[BXA]}=\frac{CE}{EA}$$
        $$\frac{[CAF-XAF]}{[CFB-XFB]}=\frac{[CAX]}{[CXB]}=\frac{AF}{FB}.$$
        Multiplying these 3 equalities gives us 
        $$\frac{BD}{DC}\cdot \frac{CE}{EA}\cdot\frac{AF}{FB} = 1.$$
        We now prove the converse. For the sake of contradiction, assume that the three segments are not concurrent.
        Name the intersection of segments BE and CF X. Now intersect AX with BC at D'. From the given we have $\frac{BD}{DC}\cdot \frac{CE}{EA}\cdot\frac{AF}{FB} = 1.$
        And since AD', BE and CF are concurrent we also have $\frac{BD'}{D'C}\cdot \frac{CE}{EA}\cdot\frac{AF}{FB} = 1.$
        This implies $\frac{BD'}{DC}=\frac{BD}{DC}$. Since D' and D are both on segment BC, they must be the same point which is a contradiction. Thus, our original assumption 
        that the three segments were not concurrent was wrong.
    \end{proof} 
 
    \begin{exercise}
        Find a trigonometric form of Ceva's Theorem.
    \end{exercise}
\section{The Centroid}
    \begin{mdframed}
        \begin{definition}
            The centroid is the intersection of the medians of a \\triangle.
       \end{definition} 
    \end{mdframed}
       
    \begin{exercise}
        Verify that the centroid always exists.
    \end{exercise}
    Often the most important fact about the centroid used in computational
    contests is that the medians divide each other into a 2:1 ratio at the centroid.
    \begin{mdframed}
        \begin{thm}
            If AD,BE, and CF are medians in $\triangle ABC$ and G is the centroid then $$\frac{AG}{GD}=\frac{BG}{GE}=\frac{CG}{GF}=2$$
        \end{thm}
    \end{mdframed}
    \begin{proof}
        Begin by drawing segment EF. We notice that $\triangle EFG \sim \triangle BFG$.
        So $\frac{BG}{GE} =2$. Repeating for the other medians finishes the proof.
    \end{proof}
    \begin{mdframed}
        \begin{corollary}
        \label{median ratios}
            The 6 triangles formed by the medians and the sides of the triangle have equal area.
        \end{corollary}
    \end{mdframed}
    \subsection*{Exercises}
    \begin{exercise}
        Prove corollary \ref{median ratios}.
    \end{exercise}
    \begin{exercise}[2018 AMC]
        Square $ABCD$ has side length $30$. Point $P$ lies inside the square so that $AP = 12$ and $BP = 26$. The centroids of $\triangle{ABP}$, $\triangle{BCP}$, $\triangle{CDP}$, and $\triangle{DAP}$ are the vertices of a convex quadrilateral. What is the area of that quadrilateral?
    \end{exercise}
    \begin{exercise} If D,E, and F are the midpoints of $\triangle ABC$ and G is the centroid, show that G is the centroid of $\triangle DEF$
    \end{exercise}
    \begin{exercise} AB is the diameter of a circle and C is a point on the circle. What shape does the centroid of $\triangle ABC$ trace as C moves around the circle.
    \end{exercise}

\section{Circumcenter}
\section{Incenter}
\section{Orthocenter}
\section{Other centers}
\section{Important Relationships}
\section{Techniques}
    \subsection{Coordinate Bashing}
    Often times, if you don't know a quick or easy way to do a geometry problem you can place the figure in the cartesian plane and work out the problem with coordinates.
    \begin{mdframed}
        \begin{thm}
            If the coordinates of a triangle are $(x_1,y_1)$,$(x_2,y_2)$, and $(x_3,y_3)$ then the centroid has coordinates $$\left(\frac{x_1+x_2+x_3}{3},\frac{y_1+y_2+y_3}{3}\right)$$
        \end{thm}
        
    \end{mdframed}
    \begin{proof}
        Let $(x_G,y_G)$ be the coordinates of the centroid and label $(x_1,y_1)$,$(x_2,y_2)$, and $(x_3,y_3)$ as A, B and C respectively.
        \\We have that the midpoint of BC is $\left(\frac{x_2+x_3}{2},\frac{y_2+y_3}{2}\right)$. So $x_G-x_1=\frac{2}{3}\cdot\frac{x_2+x_3-2x_1}{2}$.
        Therefore, $x_G = \frac{x_1+x_2+x_3}{3}$. The proof for the y component is exactly the same and we are done.
         
    \end{proof}
    \begin{exercise}[AIME 2018]
        Octagon $ABCDEFGH$ with side lengths $AB = CD = EF = GH = 10$ and $BC = DE = FG = HA = 11$ is formed by removing 6-8-10 triangles from the corners of a $23$ $\times$ $27$ rectangle with side $\overline{AH}$ on a short side of the rectangle. Let $J$ be the midpoint of $\overline{AH}$, and partition the octagon into 7 triangles by drawing segments $\overline{JB}$, $\overline{JC}$, $\overline{JD}$, $\overline{JE}$, $\overline{JF}$, and $\overline{JG}$. Find the area of the convex polygon whose vertices are the centroids of these 7 triangles.
    \end{exercise}
\end{document}