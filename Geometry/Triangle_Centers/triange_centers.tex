\documentclass[letterpaper]{article}

\usepackage{sectsty}
\usepackage{graphicx}
\usepackage{amsmath}
\usepackage{amsthm}
\usepackage{mdframed}
\usepackage{xcolor}
\usepackage{array}
\theoremstyle{plain}

\newtheorem{thm}{Theorem}[section]

\newtheorem{prop}[thm]{Proposition}
\newtheorem{corollary}{Corollary}[thm]
\newtheorem{lemma}[thm]{Lemma}

\theoremstyle{definition}

\newtheorem{definition}[thm]{Definition}
\newtheorem{example}[thm]{Example}
\newtheorem{exercise}[thm]{Exercise}

\theoremstyle{remark}
\newtheorem*{solution}{Solution}



\begin{document}

\title{Basic Triangle Centers}
\maketitle

\begin{abstract}
    Triangle centers are an important facet of euclidean geometry.
    Often problems require an intricate understanding of the relationships between triangle centers.
\end{abstract}
\section{Concurrency}
    The fact that many of the following triangles exist can be surprising. We will establish a condition which will help us prove that these centers do in fact exist.
    \begin{mdframed}
        \begin{thm}[Ceva's] 
            In $\triangle ABC$, points D, E, and F are on sides BC, AC, and AB respectively. $\frac{BD}{DC}\cdot \frac{CE}{EA}\cdot\frac{AF}{FB} = 1$, if and only if the segments AD, BE, and CF are concurrent.
            
        \end{thm}
        
    \end{mdframed} 
    \begin{proof}
        We will first show that the direct implication holds. Let the point of concurrency be X. We have the following area ratio 
        $$\frac{[ABD]}{[ADC]} = \frac{[XBD]}{[XDC]}=\frac{BD}{DC}.$$
        As you may know, this implies $$\frac{[ABD-XBD]}{[ADC-XDC]}=\frac{[ABX]}{[AXC]}=\frac{BD}{DC}.$$
        Similarly, for the other sides we have 
        $$\frac{[BCE-XCE]}{[BEA-XEA]}=\frac{[BCX]}{[BXA]}=\frac{CE}{EA}$$
        $$\frac{[CAF-XAF]}{[CFB-XFB]}=\frac{[CAX]}{[CXB]}=\frac{AF}{FB}.$$
        Multiplying these 3 equalities gives us 
        $$\frac{BD}{DC}\cdot \frac{CE}{EA}\cdot\frac{AF}{FB} = 1.$$
        We now prove the converse. For the sake of contradiction, assume that the three segments are not concurrent.
        Name the intersection of segments BE and CF X. Now intersect AX with BC at D'. From the given we have $\frac{BD}{DC}\cdot \frac{CE}{EA}\cdot\frac{AF}{FB} = 1.$
        And since AD', BE and CF are concurrent we also have $\frac{BD'}{D'C}\cdot \frac{CE}{EA}\cdot\frac{AF}{FB} = 1.$
        This implies $\frac{BD'}{DC}=\frac{BD}{DC}$. Since D' and D are both on segment BC, they must be the same point which is a contradiction. Thus, our original assumption 
        that the three segments were not concurrent was wrong.
    \end{proof} 
 
    \begin{exercise}
        Find a trigonometric form of Ceva's Theorem.
    \end{exercise}
\section{The Centroid}
    \begin{mdframed}
        \begin{definition}
            The centroid is the intersection of the medians of a \\triangle.
       \end{definition} 
    \end{mdframed}
       
    \begin{exercise}
        Verify that the centroid always exists.
    \end{exercise}
    Often the most important fact about the centroid used in computational
    contests is that the medians divide each other into a 2:1 ratio at the centroid.
    \begin{mdframed}
        If AD,BE, and FE are medians in $\triangle ABC$ and G is the centroid then $$\frac{AG}{GD}=\frac{BG}{GE}=\frac{CG}{GF}=2$$
    \end{mdframed}
    \begin{proof}
        
    \end{proof}
\end{document}