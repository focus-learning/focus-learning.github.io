\documentclass[letterpaper]{article}
\usepackage{amsmath}
\usepackage{amsthm}
\newtheorem{thm}{Theorem}[section]

\newtheorem{prop}[thm]{Proposition}
\newtheorem{corollary}{Corollary}[thm]
\newtheorem{lemma}[thm]{Lemma}
\newtheorem{example}[thm]{Example}
\newtheorem{definition}[thm]{Definition}

\theoremstyle{remark}
\newtheorem*{solution}{Solution}
\usepackage[english]{babel}
\usepackage[utf8]{inputenc}
\usepackage{graphicx}
\usepackage[colorinlistoftodos]{todonotes}

\title{Techniques for Elementary Counting}

\author{Perryn Chang}

\date{\today}
\begin{document}
\maketitle

\begin{abstract}
In this handout, we will give a basic overview of the counting techniques needed for the AMC Series. The difficulty will mostly be for the AMC 10, but may progress to the difficulty of the AIME.
\end{abstract}

\section{Introduction}

Learning to count is one of the most fundamental skills every student learns in elementary combinatorics. Therefore, we begin by learning the most common ways of counting: casework, complementary counting, and constructive counting.

\section{Fundamentals}
\label{sec:examples}

\begin{prop}
Let there be a job with $n$ parts. If the first part can be done in $a_1$ ways, the second part can be done in $a_2$ ways, and so on, such that the $n$th part can be done in $a_n$ ways, then the job can be done in $a_1 \cdot a_2 \cdot \dots \cdot a_n$ ways.

\end{prop}

\begin{example}

John's outfit consists of a hat, a shirt, and a pair of pants. If he has $5$ hats, $3$ shirts, and $2$ pants, in how many ways can John choose his outfit.

\end{example}

\begin{solution}

Here, the "job" is choosing his outfit, which consists of $3$ parts, namely choosing his hat, his shirt, and his pants. The first part of choosing his hat can be done in $5$ ways (one for each hat). Similarly, the second part of choosing his shirt can be done in $3$ ways (one for each shirt, and the last part of choosing his pants can be done in $2$ ways. Hence, the job of choosing his outfit can be done in $5\cdot 3 \cdot 2$ ways.
\end{solution}

\begin{example}

There are $10$ cards on a table, one card for each integer between $1$ and $10$. In how many ways can I choose $3$ cards with replacement such that the first card I choose is odd, the second card is prime, and the third card is a divisor of $9$.
\end{example}

\begin{solution}

Now, we have the job of choosing $3$ cards, each of which satisfies a certain condition. Although for each part we aren't given the number of ways we can complete it, we can solve for the number of ways to complete each step. For the first card, we have to choose an odd number. Between $1$ and $10$, there are $5$ odd numbers, so the first part can be done in $5$ ways. For the second part, we have to choose a prime number. There are $4$ prime numbers that we are allowed to choose, namely $2, 3, 5, $ and $7$. Therefore, there are $4$ ways to complete the second part. For the last part, there are $3$ divisors of $9$, so the last part can be completed in $3$ ways. In total, there are $5\cdot 4 \cdot 3=60$ ways to choose $3$ cards.
\end{solution}

Now we introduce a harder idea \dots



\section{Constructive Counting}

Constructive counting is perhaps one of the most intuitive and easiest methods of counting. Simply put, one counts the number of ways to construct each step of a process, and manipulates those numbers in order to find the total amount of ways to complete a process. It is a natural step from simply the fundamentals and acts as a bridge for more advanced counting techniques. In addition, constructive counting occurs frequently in both the AMC 10/12 and AIME, so it is a good idea to start with this topic.

\begin{example}
(AIME 2003)

\end{example}






\end{document}