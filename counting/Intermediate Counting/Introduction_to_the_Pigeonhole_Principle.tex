\documentclass[l1pt]{article}


\usepackage[english]{babel}
\usepackage{amsmath,amsthm,amssymb}
\usepackage{mdframed}


%% Sets page size and margins
\usepackage[margin=1in]{geometry}

%% Useful packages
\usepackage{fancyhdr}
\pagestyle{fancy}
\lhead{Introduction to the Pigeonhole Principle}
\rhead{\thepage}
\usepackage{graphicx}
\usepackage[colorlinks=true, allcolors=blue]{hyperref}
\usepackage{dirtytalk}

\theoremstyle{plain}
\newtheorem{thm}{Theorem}
\newtheorem{lemma}[thm]{Lemma}

\theoremstyle{definition}
\newtheorem{example}[thm]{Example}
\newtheorem{exercise}{Exercise}

\theoremstyle{remark}
\newtheorem*{remark}{Remark}
\newtheorem*{solution}{Solution}


\title{\textbf{Introduction to the Pigeonhole Principle}}
\author{\textbf{Perryn Chang}}
\date{\today}

\begin{document}
\maketitle

\begin{abstract}
The goal of this handout is to let the reader become familiar with the Pigeonhole Principle, an idea used in all levels of competitive mathematics. However, with that being said, this handout is targeted towards readers who have qualified for the AIME or are close to qualifying for the AIME. 
\end{abstract}

\tableofcontents

\eject

\section{Introduction}

\section{Basic Examples}

\section{More Difficult Examples}

\section{Problems}




\end{document}
