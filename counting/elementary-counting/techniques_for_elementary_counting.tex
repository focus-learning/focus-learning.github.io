%\documentclass[letterpaper]{article}
%\usepackage{amsmath}
%\usepackage{amsthm}
%\usepackage{mdframed}
%\usepackage{dirtytalk}
%\newtheorem{thm}{Theorem}[section]

%\newtheorem{prop}[thm]{Proposition}
%\newtheorem{corollary}{Corollary}[thm]
%\newtheorem{lemma}[thm]{Lemma}
%\newtheorem{example}[thm]{Example}
%\newtheorem{definition}[thm]{Definition}

%\theoremstyle{remark}
%\newtheorem*{solution}{Solution}

%\theoremstyle{definition}
%\newtheorem{exercise}[thm]{Exercise}
%\usepackage[english]{babel}
%\usepackage[utf8]{inputenc}
%\usepackage{graphicx}
%\usepackage[colorinlistoftodos]{todonotes}

%\usepackage[inline]{asymptote}

%\linespread{1.1}
%\setlength\parindent{0pt}
%\setlength{\parskip}{0.8em}

\documentclass[l1pt]{article}


\usepackage[english]{babel}
\usepackage{amsmath,amsthm,amssymb}
\usepackage{mdframed}
\usepackage{enumerate}

%\usepackage[]{microtype}
\usepackage[inline]{asymptote}



%% Sets page size and margins
\usepackage[margin=1in]{geometry}

%% Useful packages
\usepackage{fancyhdr}
\pagestyle{fancy}
\lhead{Techniques for Elementary Counting}
\rhead{\thepage}
\usepackage{graphicx}
\usepackage[colorlinks=true, allcolors=blue]{hyperref}
\usepackage{dirtytalk}

\theoremstyle{plain}
\newtheorem{thm}{Theorem}[section]
\newtheorem{prop}[thm]{Proposition}
\newtheorem{lemma}[thm]{Lemma}

\theoremstyle{definition}
\newtheorem{example}[thm]{Example}
\newtheorem{exercise}{Exercise}[section]

\theoremstyle{remark}
\newtheorem*{remark}{Remark}
\newtheorem*{solution}{Solution}


\title{\textbf{Techniques for Elementary Counting}}
\author{\textbf{Perryn Chang}}
\date{\today}




\title{Techniques for Elementary Counting}

\author{Perryn Chang}

\date{\today}

\begin{document}
\maketitle

\begin{abstract}
In this handout, we will go over fundamental ideas in counting as well as the basic counting strategies: Constructive Counting, Casework Counting, and Complementary Counting. Since there aren't a lot of theorems in introductory counting, this handout will primarily consist of key ideas encapsulated within worked problem. After the Introduction in part 1, we start with the fundamentals in part 2, and then we progress to the various counting strategies. The difficulty is intended for people who are close for qualifying for the AIME. All of the exercises in the last section are pulled from various sources so you can search them up online to find the solutions. Enjoy!
\end{abstract}

\tableofcontents

\eject

\section{Introduction}

Learning to count is one of the most fundamental skills every student learns in elementary combinatorics. We begin with the counting fundamentals and then dive into the three most common techniques for counting: Constructive Counting, Casework Counting, and Complementary Counting.

\section{Fundamentals}
\label{sec:examples}



\begin{mdframed}

\begin{prop}
Let there be a job with $n$ parts. If the first part can be done in $a_1$ ways, the second part can be done in $a_2$ ways, and so on, such that the $n$th part can be done in $a_n$ ways, then the job can be done in $a_1 \cdot a_2 \cdot \dots \cdot a_n$ ways.

\end{prop}
\end{mdframed}

\begin{example}
John's outfit consists of a hat, a shirt, and a pair of pants. If he has $5$ hats, $3$ shirts, and $2$ pants, in how many ways can John choose his outfit?
\end{example}

\begin{solution}
Here, the \say{job} is choosing his outfit, which consists of $3$ parts, namely choosing his hat, his shirt, and his pants. The first part of choosing his hat can be done in $5$ ways (one for each hat). Similarly, the second part of choosing his shirt can be done in $3$ ways (one for each shirt), and the last part of choosing his pants can be done in $2$ ways. Hence, the job of choosing his outfit can be done in $5\cdot 3 \cdot 2$ ways.
\end{solution}
\begin{example}
There are $10$ cards on a table, one card for each integer between $1$ and $10$. In how many ways can I choose $3$ cards with replacement such that the first card I choose is odd, the second card is prime, and the third card is a divisor of $9$?
\end{example}

\begin{solution}
Now, we have the job of choosing $3$ cards, each of which satisfies a certain condition. Although for each part we aren't given the number of ways we can complete it, we can solve for the number of ways to complete each step. For the first card, we have to choose an odd number. Between $1$ and $10$, there are $5$ odd numbers, so the first part can be done in $5$ ways. For the second part, we have to choose a prime number. There are $4$ prime numbers that we are allowed to choose, namely $2, 3, 5, $ and $7$. Therefore, there are $4$ ways to complete the second part. For the last part, there are $3$ divisors of $9$, so the last part can be completed in $3$ ways. In total, there are $5\cdot 4 \cdot 3=60$ ways to choose $3$ cards.
\end{solution}

\begin{example}
Freddy has $3$ different pencils that he would like to use over the course of $3$ days. In how many ways can he use his pencils?
\end{example}

\begin{solution}
We use the motivation from the previous $2$ problems. Let's think about his choices day by day. On the first day, he has $3$ pencils to choose from. On the second day, he has $2$ pencils to choose from, since he cannot choose the pencil he chose on the first day. On the third and last day, he has only $1$ pencil to choose from-the only one left. Therefore, he has $3\cdot 2\cdot 1$ ways to use $3$ pencils over the course of $3$ days.
\end{solution}

The answer to this problem is $3!$, which seems like a nice number...

\begin{example}
How many ways are there to arrange $6$ distinct children?
\end{example}

\begin{solution}
Let's think about this in terms of the position that they are arranged. For the first person in line, we have $6$ ways to choose the child. For the second person in line, we have $5$ ways to choose the child, since we cannot use the child who was already chosen to be first. Similarly, for the third person, we have $4$ ways to choose the child, since we cannot use any of the two children that were already chosen. Eventually, for the last position, there is only one way to choose the child. Therefore, the answer is $6\cdot 5\cdot 4\cdot 3\cdot 2\cdot 1=6!$, since for each spot, we have one less way to choose a child for that spot.
\end{solution}

In our previous problems, our answers were $3!$ and $6!$, which are pretty similar. We will try to generalize this.



\begin{mdframed}
\begin{prop}
The number of ways to arrange $n$ distinct items is equal to $n!$
\end{prop}
\end{mdframed}

This proposition fits Example 2.5 pretty well, but how does it relate to Example 2.4? We know that $3!$ is equal to the number of ways to arrange $3$ pencils. But, for each rearrangement, we simply choose the first pencil on day $1$, the second pencil on day $2$, and the third pencil on day $3$! Hence, each arrangement corresponds to exactly one way to use the items. In general, arranging $n$ items is the same as choosing $n$ items, one at a time, until everything is chosen.




\begin{example}
How many ways are there to rearrange the letters in \say{barb}?
\end{example}

\begin{solution}
Notice how we cannot use the proposition here, since there are $2$ b's, which are clearly not distinct. For emphasis, let the two b's be $b_1$ and $b_2$. This way we'll be able to use the proposition. It is easy to see that in this case, the answer will be $4!=24$. However, for each arrangement of $b_1$ and $b_2$, we have a similar arrangement, but with $b_1$ and $b_2$ switched. Since for the problem they are considered the same, we must divide our answer by $2$. As an example, $b_1 a b_2 r$ and $b_2 a b_1 r$ are actually the same arrangement.
\end{solution}


\begin{example}
How many ways are there to rearrange the letters in \say{juggling}?
\end{example}

\begin{solution}
We see that this problem is very similar to the previous one, except we now have $3$ repeated letters instead of $2$. However, the idea is essentially the same. If we label the g's as $g_1, g_2$, and $g_3$, for each configuration, the g's can be rearranged in $3!$ ways. Therefore, the answer is $\frac{8!}{3!}=6720$, where $8!$ is the initial count of arranging the letters, and the $3!$ exists because the g's are identical.
\end{solution}


We will generalize this crucial idea.


\begin{mdframed}
\begin{prop}
Let there be $n$ total objects and $t$ types of objects such that every object in each type is identical. Then, the number of ways to arrange these $n$ objects is the total number of ways to arrange all $n$ objects divided by the product of the number of ways to arrange each type of object. Algebraically, it is $\frac{n!}{a_1! a_2! \dots a_t!}$ where each $a_i$ is the number of objects in type $i$. 
\end{prop}
\end{mdframed}

This is pretty confusing, so lets do an example.



\begin{example}
How many ways are there to arrange $3$ red balls, $2$ blue balls, and $1$ orange ball?
\end{example}

\begin{solution}
There are $6$ total objects and $3$ types of objects. There are $3$ objects in the first type, $2$ objects in the second type, and $1$ object in the third type. Using the proposition, the answer is $\frac{6!}{3!2!1!}=60$.
\end{solution}


The next two examples demonstrate two key ideas.


\begin{example}
If there are $5$ people, how many ways are there to choose a captain and a secretary for the math team if a person cannot be both the captain and the secretary?
\end{example}

\begin{solution}
We can treat the captain as the first role and the secretary as the second role. For the captain, we have $5$ ways of choosing a person; each person is eligible. For the secretary, we have $4$ ways of choosing a person; everyone except the person who got chosen as the captain. Therefore, there are $5\cdot 4=20 $ ways of choosing a captain and a secretary.
\end{solution}


\begin{example}
If there are $5$ people, how many ways can I choose $2$ people to be part of a jury?
\end{example}

\begin{solution}
If we do the same thing as the previous problem, we end up with an answer of $20$. However, the key difference here is that the roles are the same, rather than distinct. For instance, in this problem, Ann and Bob being chosen are the same. For the previous problem, this would count as two ways, one for Ann as President and one for Bob as President. Since each time $2$ people are chosen we actually count it as one, we have to divide the answer from the previous problem by $2$. Therefore, our answer is $10$.
\end{solution}

Example 2.11 illustrates an idea known as \textbf{permutations} and Example 2.12 illustrates an idea known as \textbf{combinations}. Let's see if we can generalize them.

\bigskip


\begin{mdframed}
\begin{prop}
If we want to choose $k$ items from a total of $n$ items such that the order they are chosen matters, we have a \textbf{permutation}. The number of ways to choose them is $\frac{n!}{(n-k)}!$.
\end{prop}
\end{mdframed}

\begin{mdframed}
\begin{prop}
If we want to choose $k$ items from a total of $n$ items such that the order they are chosen does not matter, we have something known as a \textbf{combination}. The number of ways to choose the item is $\frac{n!}{(n-k)!k!}$. Typically, this is written as $\binom{n}{k}.$
\end{prop}
\end{mdframed}

Try to figure out yourself why this is true, and test these propositions with the previous examples.

\subsection*{Exercises}

\begin{exercise}
How many ways can I choose two co-captains for my math team?
\end{exercise}

\begin{exercise}
How many ways are there to arrange the letters in \say{monsoon}?
\end{exercise}

\begin{exercise}
Convince yourself that the propositions and theorems from this section are true.
\end{exercise}

\begin{exercise}
From a group of $5$ adults and $6$ children, how many ways are there to choose $3$ adults and $3$ children?
\end{exercise}






\section{Constructive Counting}

Constructive counting is perhaps one of the most intuitive and easiest methods of counting. Simply put, one counts the number of ways to construct each step of a process, and manipulates those numbers in order to find the total amount of ways to complete a process. It is a natural continuation from the fundamentals and acts as a bridge for more advanced counting techniques. In addition, constructive counting occurs frequently in both the AMC 10/12 and AIME, so we introduce this topic first.



\begin{example}
(Mathcounts Mini) A palindrome is a number that reads the same forwards and backwards. How many $7$ digit palindromes are there?

\end{example}

\begin{solution}
Since the numbers read the same forward and backwards, the first digit is the same as the last digit, the second is the same as the second to last digit, and so on. The example asks for the number of $7$ digit palindromes, so first and 7th digits are identical, the second and 6th digits are identical, the third and 5th digits are identical, and the 4th digit is by itself. Let's consider the first digit. Clearly, it can be any integer from $1$ to $9$. However, once this is decided, the 7th digit is chosen as well, since those digits must be the same. Similarly, once we have chosen the second digit, we have chosen the 6th digit as well. In this case of the second and sixth digits, we have $10$ options for the choices, since we can use $0$ here as well. For the third and fifth digits, it is the same; we have $10$ choices. For the fourth digit, we have $10$ choices as well. Therefore, the answer to the example is $9\cdot 10\cdot 10 \cdot 10=9000$, where each factor corresponds to the number of ways to choose a digit.
\end{solution}


\begin{example}
How many ways are there to arrange $2$ grandparents, $3$ fathers, and $4$ children in a row if the children insist on being next to each other?
\end{example}

\begin{solution}
Since the children will all be next to each other, we can think of them as a block of children. Now, we have $2$ grandparents, $3$ fathers, and $1$ block of children. Since all of them are distinct, there are $6!=720$ ways of arranging them. However, within the block of children, we have $4!=24$ ways to arrange the children. Therefore, our final answer is $720\cdot 24=17280.$
\end{solution}



\begin{example}
How many ways are there to arrange the letters in \say{violin} such that none of the vowels are next to each other?
\end{example}

\begin{solution}
We can begin by first placing the consonants since they have no restrictions. There are $3$ consonants so we have $3!=6$ ways to arrange them. Now we tackle the vowels. Imagine that there are $4$ spaces, $1$ before the $3$ consonants, $1$ after the consonants, and $2$ between the consonants. Clearly, these spaces are not adjacent. This means that we can choose any $3$ of them to hold our vowels. There are $\binom{4}{3}=4$ ways to choose the spaces we use, and $3!=6$ ways to arrange the $3$ vowels in those spots. Our final answer is $6\cdot 4\cdot 6=144$.
\end{solution}



\begin{example}
(2003 AIME) Define a $good~word$ as a sequence of letters that consists only of the letters $A$, $B$, and $C$ - some of these letters may not appear in the sequence - and in which $A$ is never immediately followed by $B$, $B$ is never immediately followed by $C$, and $C$ is never immediately followed by $A$. How many seven-letter good words are there?
\end{example}

\begin{solution}
Consider the first letter. We have $3$ choices for it. Then, no matter what letter we choose, we have $2$ possibilities for the next letter-any of the letters except the one that is prohibited. Similarly, no matter what letter we choose for the second letter, there are $2$ options for the third letter. This means that for all of the letters except the first one, we have $2$ options. Since we have $7$ digits in total, our answer is $3\cdot 2^6=192.$
\end{solution}


Now, we add some basic number theory to the mix. 

\begin{mdframed}
\begin{thm}
If the prime factorization of a positive integer $n$ is $p_1^{e_1}p_2^{e_2} \dots p_n^{e_n}$, then $n$ has $(e_1+1)(e_2+1)(e_3+1)\dots (e_n+1)$ factors.
\end{thm}
\end{mdframed}

We won't prove this here, but try to see why this is true. Our next problem will use a twist of this idea. The solution of it will give a sketch on why the theorem is true.
\begin{example}
How many factors of $567,000$ are perfect squares?
\end{example}

\begin{solution}
We begin by prime factorizing. The prime factorization is $2^3\cdot 3^4 \cdot 5^3 \cdot 7$. In order for a factor to be a perfect square, the exponents in its prime factorization must be even. Therefore, to create a factor of $567,000$ that is a perfect square, the exponent for $2$ must be $0$ or $2$, the exponent for $3$ must be $0,2,$ or $4$, the exponent for $5$ must be $0$ or $2$, and the exponent for $7$ must be $0$. Since we have $2$ choices for the exponent of $2$, $3$ choices for the exponent of $3$, $2$ choices for the exponent of $5$, and $1$ choice for the exponent of $7$, our answer is $2\cdot 3\cdot 2 \cdot 1=12$
\end{solution}


\subsection*{Exercises}

\begin{exercise}
(PHS TST) How many ways are there to rearrange the letters in “PRINCETON” such that no two of the vowels are
adjacent?
\end{exercise}

\begin{exercise}
(PUMaC) You have four fair $6$-sided dice, each numbered from $1$ to $6$ (inclusive). If all four dice are rolled, what is the probability that the product of the numbers rolled is prime?
\end{exercise}

\begin{exercise}
(PUMaC) A word is an ordered, non-empty sequence of letters, such as \textit{word} or \textit{wrod}. How many
distinct 3-letter words can be made from a subset of the letters c, o, m, b, o, where each letter
in the list is used no more than the number of times it appears?
\end{exercise}

\begin{exercise}
(2018 AMC) A scanning code consists of a $7 \times 7$ grid of squares, with some of its squares colored black and the rest colored white. There must be at least one square of each color in this grid of $49$ squares. A scanning code is called symmetric if its look does not change when the entire square is rotated by a multiple of $90 ^{\circ}$ counterclockwise around its center, nor when it is reflected across a line joining opposite corners or a line joining midpoints of opposite sides. What is the total number of possible symmetric scanning codes?
\end{exercise}

\section{Casework Counting}
In many cases, constructive counting is not enough to finish a problem. One often has to consider various cases and then count the number of ways to satisfy a condition for each specific case. 

\begin{example}
In how many ways can I choose $2$ positive integers that multiply to an even integer between $5$ and $10$ inclusive?
\end{example}

\begin{solution}


Let's call the two integers $a$ and $b$. Since we want $ab$ to multiply to an even integer between $5$ and $10$, $ab$ can equal $6, 8, $ or $10$. We take \textit{cases} on what $ab$ equals.
\newline

\textit{Case 1:} $ab=6:$ Here, $a$ can equal $1, 2, 3,$ or $6$. For each value of $a$, $b$ will just equal $6/a$. We have $4$ possible answers here.
\newline

\textit{Case 2:} $ab=8:$ Here, $a$ can equal $1, 2, 4,$ or $8$. For each value of $a$, $b$ will equal $8/a$. Therefore, we have $4$ possible answers.
\newline

\textit{Case 3:} $ab=10:$ Here, $a$ can equal $1, 2, 5,$ or $10$. Using the same reasoning as above, we have $4$ possible answers.
\newline

In total, we have $4(3)=12$ possibilities.
\end{solution}

\bigskip

\begin{example}
(2017 AMC) Alice refuses to sit next to either Bob or Carla. Derek refuses to sit next to Eric. How many ways are there for the five of them to sit in a row of $5$ chairs under these conditions?
\end{example}

\begin{solution}
Let the people be called $A, B, C, D$, and $E$. Since $A$ is the one with the most restrictions, we do cases on where she sits. 
\newline

\textit{Case 1:} $A$ is first: Since $A$ is first, we have $2$ options for the second person, either $D$ or $E$. For the third option, we have $2$ choices, either $B$ or $C$. For the fourth option, we have $2$ choices as well, just the two remaining options. Obviously there's only one choice for the last person, so our count here is $2^3=8$.
\newline

\textit{Case 2:} $A$ is second: The first and third positions must be occupied by $D$ and $E$, so we have $2$ possibilities here depending on the ordering of $D$ and $E$. Note that this takes care of the D/E restriction. Then, we have $2!=2$ ways of ordering $B$ and $C$ between the last two spots, so our total here is $2^2=4$.
\newline

\textit{Case 3:} $A$ is third: Since $B$ and $C$ cannot be next to $A$, $D$ and $E$ must occupy the second and fourth positions, and there are $2$ ways to place $D$ and $E$ in those positions. Then, $B$ and $C$ must occupy the first and last positions, and there are $2$ ways to place $B$ and $C$ in those positions. The total here is $2^2=4$.
\newline

Notice that when $A$ is in the fourth spot, it the number of possibilities is identical to the number of possibilities in Case 2. This is because for every case in case $2$, we can simply mirror the arrangement to get a valid seating where $A$ is in the fourth spot. From a similar argument, the number of seatings where $A$ is in the fifth spot is equal to the number of seatings where $A$ is in the first spot.


Therefore, the answer is $8\cdot 2+4\cdot 3=28$.
\end{solution}



\begin{example}
(PUMaC) What is the probability that the sum of three distinct integers between $16$ and $30$ (inclusive) is even?
\end{example}

\begin{solution}
First, note that there are $15$ integers between $16$ and $30$. We have $\binom{15}{3}=455$ ways of choosing $3$ numbers. Further, in order for $3$ integers to sum to an even number, there must be an odd amount of even numbers. In this problem, we must have either $1$ even number or $3$ even numbers. Also, there are $8$ even numbers between $16$ and $30$ and $7$ odd numbers between $16$ and $30$.
\newline

\textit{Case 1:} $1$ even number: We have $8$ ways of choosing an even number and $\binom{7}{2}=21$ ways to choose an odd number. There are $8\cdot 21=168$ ways total.
\newline

\textit{Case 2:} $3$ even numbers: We have $\binom{8}{3}=56$ ways of choosing $3$ even numbers. 
\newline

Therefore, our answer is $\frac{168+56}{455}=\frac{32}{65}$.

\end{solution}



\begin{example}
How many ways are there to distribute $10$ pieces of candy to Zhu, Chao, and Chen if the amount candy that Chao gets is a multiple of $4$?
\end{example}

\begin{solution}
Chao must get either $4$ or $8$ pieces of candy. We do casework based on how many pieces of candy he gets.
\newline

\textit{Case 1:} $4$ pieces of candy: This means that $6$ pieces of candy will be distributed between Zhu and Chen. Zhu can get between $0$ and $6$ pieces of candy, and Chen will get the rest. We have $7$ possibilities here.
\newline

\textit{Case 2:} $8$ pieces of candy: Two pieces of candy will be distributed between Zhu and Chen. Zhu can get between $0$ and $2$ pieces of candy, and Chen will take the rest. We have $3$ cases in here.
\newline

We sum up our cases to get our final answer is $7+2=9$.
\end{solution}

\subsection*{Exercises}

\begin{exercise}
(2014 AMC) Four fair six-sided dice are rolled. What is the probability that at least three of the four dice show the same value?
\end{exercise}
 
\begin{exercise}
(2015 AMC) Eight people are sitting around a circular table, each holding a fair coin. All eight people flip their coins and those who flip heads stand while those who flip tails remain seated. What is the probability that no two adjacent people will stand?
\end{exercise}

\begin{exercise}
(HMMT) A $4\times 4$ window is made out of $16$ square windowpanes. How many ways are there to stain each of the windowpanes, red, pink, or magenta, such that each windowpane is the same color as exactly two of its neighbors? Two different windowpanes are neighbors if they share a side.
\end{exercise}

\begin{exercise}
(HMMT) Let $R$ be the rectangle in the Cartesian plane with vertices at $(0,0), (2,0), (2,1),$ and $(0,1)$. $R$ can be divided into two unit squares, as shown; the resulting figure has seven edges. How many subsets of these seven edges form a connected figure?

\begin{center}

\begin{asy}
size(3cm);
draw((0,0)--(2,0)--(2,1)--(0,1)--cycle); draw((1,0)--(1,1)); 
\end{asy}

\end{center}

\end{exercise}

\section{Complementary Counting}
Complementary counting is the idea of counting the amount of things we don't want and subtracting that count from the total number of possibilities. Typically, it is used in conjunction with the previous methods in order to solve most problems. This is also the final method that we'll be going over. Almost all of the more advanced techniques use a combination of the elementary ways to count. 

Let's start with a simple example.

\begin{example}
How many $5$ digit numbers have at least $1$ one?
\end{example}

\begin{solution}
Rather than do casework based on the number of ones, we count the opposite: The number of $5$ digit numbers that have no ones. Then, we subtract this count from the total number of $5$ digit integers. Our difference is naturally the number of $5$ digit integers that have at least $1$ one. The amount of $5$ digit numbers is simply $9\cdot 10^4=90000$ by considering the digits. The amount of $5$ digit numbers without any ones is $8\cdot 9^4=52488$ by letting each digit be any integer except one. Our answer is $90000-52488=37512$.
\end{solution}



\begin{example}
(2002 AIME) Many states use a sequence of three letters followed by a sequence of three digits as their standard license-plate pattern. Given that each three-letter three-digit arrangement is equally likely, the probability that such a license plate will contain at least one palindrome (a three-letter arrangement or a three-digit arrangement that reads the same left-to-right as it does right-to-left) is $\dfrac{m}{n}$, where $m$ and $n$ are relatively prime positive integers. Find $m+n.$
\end{example}

\begin{solution}
We begin by finding the total number of arrangements without any restrictions. Since there are $26$ letters and $10$ digits, there are $26^3\cdot 10^3$ possibilities. Now, we find the number of arrangements such that there are no palindromes. Consider the letters first. We have $26$ ways to choose the first letter and $26$ ways to choose the second letter. For the last letter, we have $25$ ways to choose it, since it can't be the same as the first letter. Now let's consider the digits. There are $10$ ways each for the first and second digit. For the last digit, we have $9$ ways of choosing it, since it can't be the same as the first digit. In total, we have $26^2\cdot 25 \cdot 10^2 \cdot 9$ ways of choosing the digits such that we don't have a palindrome. The probability that we do not have a palindrome is $\frac{26^2\cdot 25\cdot 10^2 \cdot 9}{26^3\cdot 10^3}=\frac{45}{52}$. This is what we don't want, so we subtract this from one, and our desired probability is $\frac{7}{52}$. Our final answer is $59$.
\end{solution}

The next example uses an idea that is based on constructive counting. I will show the theorem and the reader will get a chance to prove it themselves.




\begin{mdframed}
\begin{thm}
Consider an $m \times n$ grid. The number of ways to get from one corner to the opposite corner in the minimal number of moves is $\binom{m+n}{n}$.
\end{thm}
\end{mdframed}


\begin{example}
Chu is at the point $(0,0)$ on the coordinate grid. How many ways can he get to to the point $(5,6)$ via the lattice points if he only goes up or right at each lattice point and wants to avoid the point $(1,4)$?
\end{example}

\begin{solution}
Without any restrictions, Chu has $\binom{11}{5}=462$ ways of getting to his destination. We now consider the cases where he has to go through the point $(1,4).$ To get from $(0,0)$ to $(1,4)$, he has $\binom{5}{4}=5$ ways to do so. Then, to get from $(1,4)$ to $(5,6)$, he has $\binom{6}{4}=15$ ways to do so. Therefore, there are $5\cdot 15=75$ ways to get from $(0,0)$ to $(5,6)$ via $(1,4)$. Since this is what we don't want, we subtract it from the final count, so our answer is $462-75=387$.
\end{solution}



\begin{example}
(2017 AMC) How many triangles with positive area have all their vertices at points $(i,j)$ in the coordinate plane, where $i$ and $j$ are integers between $1$ and $5$, inclusive?
\end{example}

\begin{solution}
Notice that any three points determine a unique triangle. The only exception to this is when the three points are collinear, so this hints at complementary counting. We start by counting the total number of (possibly-degenerate) triangles. Since there are $25$ points, there are $\binom{25}{3}=2300$ triangles in our initial count. We then do casework on where the collinear points lie.

\textit{Case 1:} $3$ points among $5$ collinear points: The rows, columns, and main diagonals have $5$ points. Since there are $5$ rows, $5$, columns, and $2$ main diagonals, and for each one we have $\binom{5}{3}=10$ ways of choosing $3$ collinear points, there are $12\cdot10=120$ degenerate triangles here.
\newline

\textit{Case 2:} $3$ points among $4$ collinear points: These consists of the diagonals adjacent to the main diagonals, and there are $4$ of them. For each of these diagonals, we can choose $\binom{4}{3}$ points, so there are $4\cdot 4=16$ degenerate triangles here.

\textit{Case 3:} $3$ points among $3$ collinear points: These consists of the diagonals parallel to the ones counted in case $2$ and the lines of slope $2,\frac{1}{2}, -\frac{1}{2}$, and $-2$. For the diagonals parallel to the ones in case $2$, there are $4$ of them, and each possibility has $1$ degenerate triangle. For the cases where the slope isn't $1$ or $-1$, there are $3$ possibilities for each of them, and each possibility has $1$ degenerate triangle. Therefore, for this case, there are $4+4\cdot3=16$ total degenerate triangles.

Since these cases are what we don't want, our answer is $2300-120-16-16=2148$.
\end{solution}

\subsection*{Exercises}
\begin{exercise}
(2006 AMC) How many four-digit positive integers have at least one digit that is a $2$ or a $3$?
\end{exercise}

\begin{exercise}
(2008 AMC) A parking lot has $16$ spaces in a row. Twelve cars arrive, each of which requires one parking space, and their drivers chose spaces at random from among the available spaces. Auntie Em then arrives in her SUV, which requires $2$ adjacent spaces. What is the probability that she is able to park?
\end{exercise}

\begin{exercise}
(HMMT) Find the number of ordered pairs of integers $(a, b)$ such that $a, b$ are divisors of $720$ but $ab$ is not.
\end{exercise}

\section{Problems}
This section will contain problems from various competitions. They are arranged \textit{roughly} in order. Problems near the end will be especially tricky.

\begin{exercise}
(PUMaC) How many ways can you arrange $3$ Alice’s, $1$ Bob, $3$ Chad’s, and $1$ David in a line if the Alice’s are all indistinguishable, the Chad’s are all indistinguishable, and Bob and David want to be adjacent to each other? (In other words, how many ways can you arrange $3$ A’s, $1$ B, $3$ C’s,and $1$ D in a row where the B and D are adjacent?)
\end{exercise}

\begin{exercise}
(P\_Groudon January Mock AMC) How many ways can 2 $As$, 2 $Bs$, 1 $C$, 1 $D$, and 1 $E$ be arranged in a line such that the two $As$ are next to each other or the two $Bs$ are next to each other, but not both at the same time?
\end{exercise}

\begin{exercise}
(HMMT) How many distinct permutations of the letters of the word REDDER are there that do not contain a palindromic substring of length at least two? (A substring is a contiguous block of letters that is part of the string. A string is palindromic if it is the same when read backwards.)
\end{exercise}

\begin{exercise}
(AOIME) Find the number of ordered pairs of positive integers $(m,n)$ such that ${m^2n = 20 ^{20}}$.
\end{exercise}

\begin{exercise}
(PUMaC) There are five dots arranged in a line from left to right. Each of the dots is colored from one of five colors so that no $3$ consecutive dots are all the same color. How many ways are there to color the dots?
\end{exercise}

\begin{exercise}
(2019 AMC) How many positive integer divisors of $201^9$ are perfect squares or perfect cubes (or both)?
\end{exercise}

\begin{exercise}
(PUMaC) Chitoge is painting a cube; she can paint each face either black or white, but she wants no vertex of the cube to be touching three faces of the same color. In how many ways can Chitoge paint the cube? Two paintings of a cube are considered to be the same if you can rotate one cube so that it looks like the other cube.
\end{exercise}

\begin{exercise}
(2018 AMC) Three young brother-sister pairs from different families need to take a trip in a van. These six children will occupy the second and third rows in the van, each of which has three seats. To avoid disruptions, siblings may not sit right next to each other in the same row, and no child may sit directly in front of his or her sibling. How many seating arrangements are possible for this trip?
\end{exercise}

\begin{exercise}
(2017 AMC) Three fair six-sided dice are rolled. What is the probability that the values shown on two of the dice sum to the value shown on the remaining die?
\end{exercise}

\begin{exercise}
(2019 AMC) A child builds towers using identically shaped cubes of different color. How many different towers with a height $8$ cubes can the child build with $2$ red cubes, $3$ blue cubes, and $4$ green cubes? (One cube will be left out.)
\end{exercise}

\begin{exercise}
(2018 AMC) A number $m$ is randomly selected from the set $\{11,13,15,17,19\}$, and a number $n$ is randomly selected from $\{1999,2000,2001,\newline \ldots,2018\}$. What is the probability that $m^n$ has a units digit of $1$?
\end{exercise}

\begin{exercise}
(2012 AMC) A $3\times3$ square is partitioned into $9$ unit squares. Each unit square is painted either white or black with each color being equally likely, chosen independently and at random. The square is then rotated $90^\circ$ clockwise about its center, and every white square in a position formerly occupied by a black square is painted black. The colors of all other squares are left unchanged. How many ways are there to initially paint the grid such that the grid is now black?
\end{exercise}

\begin{exercise}
(HMMT) There are $10$ people who want to choose a committee of $5$ people among them. They do this by first electing a set of $1, 2, 3$, or $4$ committee leaders, who then choose among the remaining people to complete the $5$-person committee. In how many ways can the committee be formed, assuming that people are distinguishable? (Two committees that have the same members but different sets of leaders are considered to be distinct.)
\end{exercise}

\begin{exercise}
(HMMT) Given an $8\times 8$ checkerboard with alternating white and black squares, how many ways are there to choose four black squares and four white squares so that no two of the eight chosen squares are in the same row or column?
\end{exercise}


\begin{exercise}
(OTSS Season 1) Let $(a \oplus b)$ denote the bitwise exclusive-or (XOR) of $a$ and $b$. This is equivalent to adding $a$ and $b$ in binary (base-two), but discarding the \say{carry} to the next place value if it is applicable. For instance, $(1_2 \oplus 1_2) = 0_2$, $(1_2 \oplus 0_2) = 1_2$, and $(5\oplus 3)= (101_2\oplus 011_2)= 110_2$. How many ordered pairs of nonnegative integers $(x,y)$ both less than $32$ satisfy $(x \oplus y) > x \geq y$?
\end{exercise}

\begin{exercise}
(2005 AIME) Robert has $4$ indistinguishable gold coins and $4$ indistinguishable silver coins. Each coin has an engraving of one face on one side, but not on the other. He wants to stack the eight coins on a table into a single stack so that no two adjacent coins are face to face. Find the number of possible distinguishable arrangements of the $8$ coins.
\end{exercise}

\begin{exercise}
(2004 AIME) How many positive integers less than $10,000$ have at most two different digits?
\end{exercise}

\begin{exercise}
(2020 AIME) Six cards numbered $1$ through $6$ are to be lined up in a row. Find the number of arrangements of these six cards where one of the cards can be removed leaving the remaining five cards in either ascending or descending order.
\end{exercise}


\end{document}
