\documentclass[letterpaper]{article}
\usepackage{amsmath}
\usepackage{amsthm}
\newtheorem{thm}{Theorem}[section]

\newtheorem{prop}[thm]{Proposition}
\newtheorem{corollary}{Corollary}[thm]
\newtheorem{lemma}[thm]{Lemma}
\newtheorem{example}[thm]{Example}
\newtheorem{definition}[thm]{Definition}

\theoremstyle{remark}
\newtheorem*{solution}{Solution}

\theoremstyle{definition}
\newtheorem{exercise}[thm]{Exercise}
\usepackage[english]{babel}
\usepackage[utf8]{inputenc}
\usepackage{graphicx}
\usepackage[colorinlistoftodos]{todonotes}

\usepackage[inline]{asymptote}

\title{Techniques for Elementary Counting}

\author{Perryn Chang}

\date{\today}

\begin{document}
\maketitle

\begin{abstract}
In this handout, we will give a basic overview of the counting techniques needed for the AMC Series. The difficulty will mostly be for the AMC 10, but may progress to the difficulty of the AIME.
\end{abstract}

\section{Introduction}

Learning to count is one of the most fundamental skills every student learns in elementary combinatorics. Therefore, we begin by learning the most common ways of counting: casework, complementary counting, and constructive counting.

\section{Fundamentals}
\label{sec:examples}

\begin{prop}
Let there be a job with $n$ parts. If the first part can be done in $a_1$ ways, the second part can be done in $a_2$ ways, and so on, such that the $n$th part can be done in $a_n$ ways, then the job can be done in $a_1 \cdot a_2 \cdot \dots \cdot a_n$ ways.

\end{prop}

\begin{example}

John's outfit consists of a hat, a shirt, and a pair of pants. If he has $5$ hats, $3$ shirts, and $2$ pants, in how many ways can John choose his outfit?

\end{example}

\begin{solution}

Here, the "job" is choosing his outfit, which consists of $3$ parts, namely choosing his hat, his shirt, and his pants. The first part of choosing his hat can be done in $5$ ways (one for each hat). Similarly, the second part of choosing his shirt can be done in $3$ ways (one for each shirt, and the last part of choosing his pants can be done in $2$ ways. Hence, the job of choosing his outfit can be done in $5\cdot 3 \cdot 2$ ways.
\end{solution}

\begin{example}

There are $10$ cards on a table, one card for each integer between $1$ and $10$. In how many ways can I choose $3$ cards with replacement such that the first card I choose is odd, the second card is prime, and the third card is a divisor of $9$?
\end{example}

\begin{solution}

Now, we have the job of choosing $3$ cards, each of which satisfies a certain condition. Although for each part we aren't given the number of ways we can complete it, we can solve for the number of ways to complete each step. For the first card, we have to choose an odd number. Between $1$ and $10$, there are $5$ odd numbers, so the first part can be done in $5$ ways. For the second part, we have to choose a prime number. There are $4$ prime numbers that we are allowed to choose, namely $2, 3, 5, $ and $7$. Therefore, there are $4$ ways to complete the second part. For the last part, there are $3$ divisors of $9$, so the last part can be completed in $3$ ways. In total, there are $5\cdot 4 \cdot 3=60$ ways to choose $3$ cards.
\end{solution}


\begin{example}
Freddy has $3$ different pencils that he would like to use over the course of $3$ days. In how many ways can he use his pencils?

\end{example}

\begin{solution}
We use the motivation from the previous $2$ problems. Let's think about his choices day by day. On the first day, he has $3$ pencils to choose from. On the second day, he has $2$ pencils to choose from, since he cannot choose the pencil he chose on the first day. On the third and last day, he has only $1$ pencil to choose from-the only one left. Therefore, he has $3\cdot 2\cdot 1$ ways to use $3$ pencils over the course of $3$ days.
\end{solution}



Hmmmm, the answer to this problem is $3!$, which seems like a "nice" number. 

\begin{example}

How many ways are there to arrange $6$ distinct children?
\end{example}

\begin{solution}

Let's think about this in terms of the position that they are arranged. For the first person in line, we have $6$ ways to choose the child. For the second person in line, we have $5$ ways to choose the child, since we cannot use the child who was already chosen to be first. Similarly, for the third person, we have $4$ ways to choose the child, since we cannot use any of the two children that were already chosen. Eventually, for the last position, there is only one way to choose the child. Therefore, the answer is $6\cdot 5\cdot 4\cdot 3\cdot 2\cdot 1=6!$, since for each spot, we have one less way to choose a child for that spot.
\end{solution}



In our previous problems, our answers were $3!$ and $6!$, which are pretty similar, except with the $3$ and $6$ switched. We will try to generalize this.

\begin{prop}
The number of ways to arrange $n$ distinct items is equal to $n!$
\end{prop}

This proposition fits Example 2.5 pretty well, but how does it relate to Example 2.4? We know that $3!$ is equal to the number of ways to rearrange $3$ pencils. But, for each rearrangement, we simply choose the first pencil on day $1$, the second pencil on day $2$, and the third pencil on day $3$! Hence, each arrangement corresponds to exactly one way to use the items. In general, arranging $n$ items is the same as choosing $n$ items, then choosing $n-1$ items, then $n-2$ items, and so on , until one item it chosen. Convince yourself that this is indeed correct. 
\newline

\begin{example}
How many ways are there to rearrange the letters in "barb"?
\end{example}

\begin{solution}
Notice how we cannot use the proposition here, since there are $2$ b's, which are clearly not distinct. For emphasis, let the two b's be $b_1$ and $b_2$. This way we'll be able to use the proposition. It is easy to see that in this case, the answer will be $4!=24$. However, for each arrangement of $b_1$ and $b_2$, we have a similar arrangement, but with $b_1$ and $b_2$ switched. Since for the problem they are considered the same, we must divide our answer by $2$. As an example, $b_1 a b_2 r$ and $b_2 a b_1 r$ are actually the same arrangement.
\end{solution}

\begin{example}
How many ways are there to rearrange the letters in "juggling"?
\end{example}

\begin{solution}
We see that this problem is very similar to the previous one, except we now have $3$ repeated letters instead of $2$. However, the idea is essentially the same. If we label the g's as $g_1, g_2$, and $g_3$, for each configuration, the g's can be rearranged in $3!$ ways. Therefore, the answer is $\frac{8!}{3!}=6720$, where $8!$ is the initial count of arranging the letters, and the $3!$ exists because the g's are identical.
\end{solution}

\begin{thm}
Placeholder
\end{thm}

The next two examples demonstrate two key ideas.

\begin{example}
If there are $5$ people, how many ways are there to choose a captain and a secretary for the math team if a person cannot be both the captain and the secretary?
\end{example}

\begin{solution}
We can treat the captain as the first role and the secretary as the second role. For the captain, we have $5$ ways of choosing a person; each person is eligible. For the secretary, we have $4$ ways of choosing a person; everyone except the person who got chosen as the captain. Therefore, there are $5\cdot 4=20 $ ways of choosing a captain and a secretary.
\end{solution}

\begin{example}
If there are $5$ people, how many ways can I choose $2$ people to be part of a jury?
\end{example}

\begin{solution}
If we do the same thing as the previous problem, we end up with an answer of $20$. However, the key difference here is that the roles are the same, rather than distinct. For instance, in this problem, Ann and Bob being chosen are the same. For the previous problem, this would count as two ways, one for Ann as President and one for Bob as President. Since each time $2$ people are chosen we actually count it as one, we have to divide the answer from the previous problem by $2$. Therefore, our answer is $10$.
\end{solution}

Example 2.10 illustrates an idea known as \textbf{permutations} and Example 2.11 illustrates an idea known as \textbf{combinations}. Let's see if we can generalize them.

\begin{prop}
In simple language, if we want to choose $k$ items from a total of $n$ items such that the order they are chosen matters, we have a \textbf{permutation}. The number of ways to choose them is $\frac{n!}{(n-k)}!$.
\end{prop}

\begin{prop}
If we want to choose $k$ items from a total of $n$ items such that the order they are chosen does not matter, we have something known as a \textbf{combination}. The number of ways to choose the item is $\frac{n!}{(n-k)!k!}$. Typically, this is written as $\binom{n}{k}.$
\end{prop}

Try to figure out yourself why this is true, and test these propositions with the previous examples.

\subsection*{Exercises}

\begin{exercise}
How many ways can I choose two co-captains for my math team?
\end{exercise}

\begin{exercise}
How many ways are there to arrange the letters in "monsoon"?
\end{exercise}

\begin{exercise}
Convince yourself that the propositions and theorems from this section are true.
\end{exercise}

\begin{exercise}
From a group of $5$ adults and $6$ children, how many ways are there to choose $3$ adults and $3$ children?
\end{exercise}






\section{Constructive Counting}

Constructive counting is perhaps one of the most intuitive and easiest methods of counting. Simply put, one counts the number of ways to construct each step of a process, and manipulates those numbers in order to find the total amount of ways to complete a process. It is a natural step from simply the fundamentals and acts as a bridge for more advanced counting techniques. In addition, constructive counting occurs frequently in both the AMC 10/12 and AIME, so it is a good idea to start with this topic.

\begin{example}
(Mathcounts Mini) A palindrome is a number that reads the same forwards and backwards. How many $7$ digit palindromes are there?

\end{example}

\begin{solution}
Since the numbers read the same forward and backwards, the first digit is the same as the last digit, the second is the same as the second to last digit, and so on. The example asks for the number of $7$ digit palindromes, so first and 7th digits are identical, the second and 6th digits are identical, the third and 5th digits are identical, and the 4th digit is by itself. Let's consider the first digit. Clearly, it can be any integer from $1$ to $9$. However, once this is decided, the 7th digit is chosen as well, since those digits must be the same. Similarly, once we have chosen the second digit, we have chosen the 6th digit as well. In this case, we have $10$ options for the choices, since we can use $0$ here as well. For the third and fifth digits, it is the same; we have $10$ choices. For the fourth digit, we have $10$ choices as well. Therefore, the answer to the example is $9\cdot 10\cdot 10 \cdot 10=9000$, where each factor corresponds to the number of ways to choose a digit.
\end{solution}

\begin{example}
How many ways are there to arrange $2$ grandparents, $3$ fathers, and $4$ children in a row if the children insist on being next to each other?
\end{example}

\begin{solution}
Since the children will all be next to each other, we can think of them as a block. Now, we have $2$ grandparents, $3$ fathers, and $1$ block of children. Since all of them are distinct, there are $6!=720$ ways of arranging them. However, within the block of children, we have $4!=24$ ways to arrange the children. Therefore, our final answer is $720\cdot 24=17280.$
\end{solution}

\begin{example}
How many ways are there to arrange the letters in "violin" such that none of the vowels are next to each other?
\end{example}

\begin{solution}
We can begin by first placing the consonants since they have no restrictions. There are $3$ consonants so we have $3!=6$ ways to arrange them. Now we tackle the vowels. 
Imagine that there are $4$ spaces, $2$ before and after the $3$ consonants, and $2$ between the consonants. Clearly, these spaces are not adjacent. This means that we can choose any $3$ of them to hold our vowels. There are $\binom{4}{3}=4$ ways to choose the spaces we use, and $3!=6$ ways to arrange the $3$ vowels in those spots. Our final answer is $6\cdot 4\cdot 6=144$.
\end{solution}

\begin{example}
(2003 AIME) Define a $good~word$ as a sequence of letters that consists only of the letters $A$, $B$, and $C$ - some of these letters may not appear in the sequence - and in which $A$ is never immediately followed by $B$, $B$ is never immediately followed by $C$, and $C$ is never immediately followed by $A$. How many seven-letter good words are there?
\end{example}

\begin{solution}
Consider the first letter. We have $3$ choices for it. Then, no matter what letter we choose, we have $2$ possibilities for the next letter. Similarly, no matter what letter we choose for the second letter, there are $2$ options for the third letter. This means that for all of the letters except the first one, we have $2$ options. Since we have $7$ digits in total, our answer is $3\cdot 2^6=192.$
\end{solution}


Now, we add some basic number theory to the mix. 

\begin{thm}
If the prime factorization of a positive integer $n$ is $p_1^{e_1}p_2^{e_2} \dots p_n^{e_n}$, then $n$ has $(e_1+1)(e_2+1)(e_3+1)\dots (e_n+1)$ factors.

\end{thm}

We won't prove this here, but try to see why this is true. Our next problem will use a twist of this idea. The solution of it will give a sketch on why the theorem is true.
\begin{example}
How many factors of $567,000$ are perfect squares?
\end{example}

\begin{solution}
We begin by prime factorizing. The prime factorization is $2^3\cdot 3^4 \cdot 5^3 \cdot 7$. In order for a factor to be a perfect square, the exponents in its prime factorization must be even. Therefore, to create a factor of $567,000$ that is a perfect square, the exponent for $2$ must be $0$ or $2$, the exponent for $3$ must be $0,2,$ or $4$, the exponent for $5$ must be $0$ or $2$, and the exponent for $7$ must be $0$. Since we have $2$ choices for the exponent of $2$, $3$ choices for the exponent of $3$, $2$ choices for the exponent of $5$, and $1$ choice for the exponent of $7$, our answer is $2\cdot 3\cdot 2 \cdot 1=12$
\end{solution}

\begin{example}
(2012 AMC) A $3\times3$ square is partitioned into $9$ unit squares. Each unit square is painted either white or black with each color being equally likely, chosen independently and at random. The square is then rotated $90^\circ$ clockwise about its center, and every white square in a position formerly occupied by a black square is painted black. The colors of all other squares are left unchanged. How many ways are there to initially paint the grid such that the grid is now black?

\end{example}

\begin{solution}


First, notice that the center square must be black. If it was white, a $90^\circ$ rotation would leave it at the same position, so it would still be white. Then, notice that the four corner squares and the four middle squares do not affect each other. A rotation puts a corner square in another corner square and a middle square in a middle square. This means that we can consider the corner and middle squares separately. Let's do the corner squares first. If they are all black, then it is easy to see that all the corner squares would be black after a rotation. Here, we only have $1$ way to color the square. If $3$ of the corner squares are colored black, then a rotation would make all the squares black. This is because after the rotation, the sole white square is forced to land on a square that was previously black. We have $4$ ways to color the corners with $3$ blacks squares. If only $2$ squares are black, then notice that we must choose the corners that are diagonal from each other. If not, one white square will go into a spot that was previously white, so it will not be colored black. However, with the diagonals being colored black, each white square will go to a square that was previously black. There are $2$ ways to color the corners with $2$ black squares. It is easy to see that it is impossible to color the corners with $1$ black square, so in total, there are $1+4+2=7$ ways to color the corners. With a similar argument, we have $7$ ways to color the middle squares, so our final answer is $1\cdot 7 \cdot 7=49$.
\end{solution}

\subsection*{Exercises}

\begin{exercise}
(PHS TST) How many ways are there to rearrange the letters in “PRINCETON” such that no two of the vowels are
adjacent?
\end{exercise}

\begin{exercise}
(PUMaC) You have four fair $6$-sided dice, each numbered from $1$ to $6$ (inclusive). If all four dice are rolled, what is the probability that the product of the numbers rolled is prime?
\end{exercise}

\begin{exercise}
(PUMaC) A word is an ordered, non-empty sequence of letters, such as \textit{word} or \textit{wrod}. How many
distinct 3-letter words can be made from a subset of the letters c, o, m, b, o, where each letter
in the list is used no more than the number of times it appears?
\end{exercise}

\begin{exercise}
(HMMT) Given an $8\times 8$ checkerboard with alternating white and black squares, how many ways are there to choose four black squares and four white squares so that no two of the eight chosen squares are in the same row or column?
\end{exercise}

\begin{exercise}
(2018 AMC) A scanning code consists of a $7 \times 7$ grid of squares, with some of its squares colored black and the rest colored white. There must be at least one square of each color in this grid of $49$ squares. A scanning code is called symmetric if its look does not change when the entire square is rotated by a multiple of $90 ^{\circ}$ counterclockwise around its center, nor when it is reflected across a line joining opposite corners or a line joining midpoints of opposite sides. What is the total number of possible symmetric scanning codes?
\end{exercise}

\section{Casework Counting}
In many cases, constructive counting is not enough to finish a problem. One often has to consider various cases and then count the number of ways to satisfy a condition for each specific case. In the last example of the previous section, we looked over a basic example of casework by considering the amount of black squares that were initially colored. We now begin to explore casework in more depth.

\begin{example}
In how many ways can I choose $2$ positive integers that multiply to an even integer between $5$ and $10$ inclusive?
\end{example}

\begin{solution}


Let's call the two integers $a$ and $b$. Since we want $ab$ to multiply to an even integer between $5$ and $10$, $ab$ can equal $6, 8, $ or $10$. We take \textit{cases} on what $ab$ equals.
\newline

\textit{Case 1:} $ab=6:$ Here, $a$ can equal $1, 2, 3,$ or $6$. For each value of $a$, $b$ will just equal $6/a$. We can $6$ possible answers here.
\newline

\textit{Case 2:} $ab=8:$ Here, $a$ can equal $1, 2, 4,$ or $8$. For each value of $a$, $b$ will equal $8/a$. Therefore, we have $4$ possible answers.
\newline
\textit{Case 3:} $ab=10:$ Here, $a$ can equal $1, 2, 5,$ or $10$. Using the same reasoning as above, we have $4$ possible answers.
\newline

In total, we have $4(3)=12$ possibilities.
\end{solution}

\begin{example}
(2017 AMC) Alice refuses to sit next to either Bob or Carla. Derek refuses to sit next to Eric. How many ways are there for the five of them to sit in a row of $5$ chairs under these conditions?
\end{example}

\begin{solution}
Let the people be called $A, B, C, D$, and $E$. Since $A$ is the one with the most restrictions, we do cases on where she sits. 
\newline

\textit{Case 1:} $A$ is first: Since $A$ is first, we have $2$ options for the second person, either $D$ or $E$. For the third option, we have $2$ choices, either $B$ or $C$. For the fourth option, we have $2$ choices as well, just the two remaining options. Obviously there's only one choice for the last person, so our count here is $2^3=8$.
\newline

\textit{Case 2:} $A$ is second: The first and third positions must be occupied by $D$ and $E$, so we have $2$ possibilities here depending on the ordering of $D$ and $E$. Then, we have $2!=2$ ways of ordering $B$ and $C$ between the last two spots, so our total here is $2^2=4$.
\newline

\textit{Case 3:} $A$ is third: $D$ and $E$ must occupy the second and fourth positions, and there are $2$ ways to place $D$ and $E$ in those positions. Then, $B$ and $C$ must occupy the first and last positions, and there are $2$ ways to place $B$ and $C$ in those positions. The total here is $2^2=4$.
\newline

Notice that when $A$ is in the fourth spot, it the number of possibilities is identical to the number of possibilities in Case 2. This is because for every case in case $2$, we can simply mirror the arrangement to get a valid seating where $A$ is in the fourth spot. From a similar argument, the number of seatings where $A$ is in the fifth spot is equal to the number of seatings where $A$ is in the first spot.


Therefore, the answer is $8\cdot 2+4\cdot 3=28$.
\end{solution}

\begin{example}
(PUMaC) What is the probability that the sum of three distinct integers between $16$ and $30$ (inclusive) is even?
\end{example}

\begin{solution}
First, note that there are $15$ integers between $16$ and $30$. We have $binom{15}{3}=455$ ways of choosing $3$ numbers. Further, in order for $3$ integers to sum to an even number, there must be an odd amount of even numbers. In this problem, we must have either $1$ even number or $3$ even numbers. Also, there are $8$ even numbers between $16$ and $30$ and $7$ odd numbers between $16$ and $30$.
\newline

\textit{Case 1:} $1$ even number: We have $8$ ways of choosing an even number and $\binom{7}{2}=21$ ways to choose an odd number. There are $8\cdot 21=168$ ways total.
\newline

\textit{Case 2:} $3$ even numbers: We have $\binom{8}{3}=56$ ways of choosing $3$ even numbers. 
\newline

Therefore, our answer is $\frac{168+56}{455}=\frac{32}{65}$.

\end{solution}

\begin{example}
Add self created example later
\end{example}

\subsection*{Exercises}

\begin{exercise}
(2019 AMC) For a set of four distinct lines in a plane, there are exactly $N$ distinct points that lie on two or more of the lines. What is the sum of all possible values of $N$?
\end{exercise}
 
\begin{exercise}
(2015 AMC) Eight people are sitting around a circular table, each holding a fair coin. All eight people flip their coins and those who flip heads stand while those who flip tails remain seated. What is the probability that no two adjacent people will stand?
\end{exercise}

\begin{exercise}
(HMMT) A $4\times 4$ window is made out of $16$ square windowpanes. How many ways are there to stain each of the windowpanes, red, pink, or magenta, such that each windowpane is the same color as exactly two of its neighbors? Two different windowpanes are neighbors if they share a side.
\end{exercise}

\begin{exercise}
(HMMT) Let $R$ be the rectangle in the Cartesian plane with vertices at $(0,0), (2,0), (2,1),$ and $(0,1)$. $R$ can be divided into two unit squares, as shown; the resulting figure has seven edges. How many subsets of these seven edges form a connected figure?

\begin{center}

\begin{asy}
size(3cm);
draw((0,0)--(2,0)--(2,1)--(0,1)--cycle); draw((1,0)--(1,1)); 
\end{asy}

\end{center}

\end{exercise}

\section{Complementary Counting}
Complementary counting is the idea of counting the amount of things we don't want and subtracting that count from the total number of possibilities. Typically, it is used in conjunction with the previous methods in order to solve most problems. This is also the final method that we'll be going over. Almost all of the more advanced techniques use a combination of the elementary ways to count. 

\end{document}
