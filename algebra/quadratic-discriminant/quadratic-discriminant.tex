\documentclass{article}
\usepackage[utf8]{inputenc}
\usepackage{mdframed}
\usepackage{amssymb}

\title{The Quadratic Discriminant}
\author{Alexander Chen}
\date{July 1, 2020}

\begin{document}

\maketitle

\section{Introduction}
The discriminant is a crucial tool in solving problems related to quadratics, or polynomials of degree two.\\\\

Recall that the general form of a quadratic is $ax^2+bx+c=0$ with $a\neq0$. To solve for the discriminant, we first divide both sides by $a$. $$x^2+\frac{b}{a}x+\frac{c}{a}=0$$
Adding $(\frac{b}{2a})^2$, or $\frac{b^2}{4a^2}$, to both sides of the equation allows us to complete the square.
$$x^2+\frac{b}{a}x+\frac{b^2}{4a^2}+\frac{c}{a}=\frac{b^2}{4a^2}$$
$$(x+\frac{b}{2a})^2=\frac{b^2}{4a^2}-\frac{c}{a}$$
We arrive at the key equation:
$$(x+\frac{b}{2a})^2=\frac{b^2-4ac}{4a^2}$$\\

\textbf{Definition 1.1.} The discriminant of a quadratic in the form $ax^2+bx+c=0$ is the number $b^2-4ac$.\\

We know that a number $c \in \mathbb{R}$ ($c$ is a real number) has $c^2>0$. However, if $c^2<0$, $c \in \mathbb{C}$ and $c \not\in \mathbb{R}$ ($c$ is a non-real complex number). The left-hand side of the key equation, $(x+\frac{b}{2a})^2$, follows the same rules. Note that since $4a^2 > 0$, the discriminant is the deciding factor for whether $(x+\frac{b}{2a})^2 \in \mathbb{R}$ or not. We get the following theorem:\\

\begin{mdframed}
    \textbf{Theorem 1.2.}
    \begin{itemize}
        \item If the discriminant is less than 0, then $ax^2+bx+c$ does not have real roots.
        \item If the discriminant is equal to 0, then $ax^2+bx+x$ has one real root.
        \item If the discriminant is greater than 0, then $ax^2+bx+c$ has two real roots.
    \end{itemize}
\end{mdframed}

\textbf{Exercise 1.3} Check if $5x^2+6x+8$ has real roots and if so, the number of real roots.


\end{document}
